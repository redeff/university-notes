\documentclass{article}
\usepackage{amssymb}
\usepackage{mathtools}

\everymath{\displaystyle}
\setlength{\parskip}{3mm}
\setlength{\parindent}{0mm}

\def\R{\mathbb{R}}
\def\C{\mathbb{C}}
\def\N{\mathbb{N}}
\def\Q{\mathbb{Q}}

\date{}
\author{}

\begin{document}
\section*{Ejercicio 1}
Dado $f : \K_2[x] \to \K_2[x]$, y $fp = p' + x^2p''$.
Calcular $\Ima f$, $\Nu f$, y $[f]_E$, y $f^{-1} (2x^2 + 4x -1)$.

Notemos que $
[f]_E = 
\begin{bmatrix}
	0 & 1 & 0 \\
	0 & 0 & 2 \\
	0 & 0 & 1 \\
\end{bmatrix}
$

Notemos que $\Ima [f]$ es el generado por las columnas, luego para sacar la
$\Ima f$, lo evaluamos en la canónica y estamos.

Para calcular la preimagen de $4x^2 + 4x - 1$, traduciéndolo a $\K^3$,
hay que resolver el sistema:

\[
\begin{bmatrix}
	0 & 1 & 0 \\
	0 & 0 & 2 \\
	0 & 0 & 1 \\
\end{bmatrix} x = 
\begin{bmatrix}
	-1 \\ 4 \\ 4
\end{bmatrix}
\]

Que si lo resolvemos tenemos que $x \in (\Nu [f] = \langle (1 \; 0\; 0) \rangle
)+ (0 \; -1 \; 2)$.
Para sacarlo en $\K_2[x]$, tenemos que son los de la pinta $px = \lambda - x +
2x^2$.

\section*{Matrices equivalentes y semejantes}
Decidir si, dados
$
A = 
\begin{bmatrix}
	-4 & -6 \\
	2 & 3 \\
\end{bmatrix}
\quad
B = 
\begin{bmatrix}
	4 & 3 \\
	1 & 2 \\
\end{bmatrix}
$, tenemos $A \sim B$.

Notemos que $\rg A = 1$, y $\rg B = 2$, entonces no se puede.

\section*{Ejercicio 3}
Dados
$
A=
\begin{bmatrix}
	7&1&2\\
	0 & 3 & -1 \\
	-3 & 4 & -2 \\
\end{bmatrix}
\quad
B=
\begin{bmatrix}
	1&0&1\\
	0 & 2 & 1 \\
	1 & 1 & 1 \\
\end{bmatrix}
$

decidir si $A \sim B$, y si existen $\tl f$ tal que $[f]_\BB = A$ y que
$[f]_{\BB'\BB''} = B$.

\section*{Condiciones necesarias para Semejanza}
Si dos matrices $A, B$ tienen $A \sim B$, luego $\rg A = \rg B$, y $\tr A = \tr
B$. Notemos que la vuelta no vale!

\section*{Matrices Equivalentes}
Dos matrices $A, B \in \K^{n \times m}$ son equivalentes sii $\exists C \in
\GL \; \K^n, D \in \GL \; \K^m$ tal que $CAD = B$.

Además, si $A, B$ inversibles, notemos que:
$
A^{-1}AB = B
$, entonces todas las inversibles son equivalentes.

\section*{Ejercicio 4}
\subsection*{Parte a}
Sea $A \in \K^{n \times m}, B \in \GL \; \K^n$. Probar que $\rg BA = \rg A$, y
lo mismo para el otro lado.

Si tomamos $f x = Ax, gx = Bx$, luego como $B$ inversible notemos $\iso g$,
luego $\dim f \K^m = \dim g (f \K^m) = \dim \Ima (g \circ f)$

\subsection*{Parte b}
Sea $A \in \R^{4 \times 4}$ de rango $2$, demostrar que existen $B \in \R^{4
\times 2}, C \in \R^{2 \times 4}$ tales que $A = BC$

Tomamos $\BB = \{b_1, b_2, b_3, b_4\}$ base de $\R^4$, conde $\langle b_1, b_2
\rangle = \Nu A$.

Luego tomemos $g b_1 = g b_2 = 0$, y $g b_3 = (1, 0)$, $g b_4 = (0, 1)$, luego
tomamos $h (1, 0) = f b_3$, $h (0, 1) = f b_4$. Notemos que $h \circ g$
coincide con $f$ en la base $\BB$, luego ya estamos.

Alternativamente, podemos tomar $h = f|^{\Ima f}$, y $g = \id|_{\Ima f}$, luego
tenemos que $f = g \circ h$
\end{document}
