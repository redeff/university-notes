\documentclass{article}
\usepackage{amssymb}
\usepackage{mathtools}

\everymath{\displaystyle}
\setlength{\parskip}{3mm}
\setlength{\parindent}{0mm}

\def\R{\mathbb{R}}
\def\C{\mathbb{C}}
\def\N{\mathbb{N}}
\def\Q{\mathbb{Q}}

\date{}
\author{}

\title{Segunda Práctica}
\begin{document}
	\maketitle
	Dado $S \in \V$, $\langle S \rangle$ es la combinación.
	Se dice que $S$ es un sistema de generadores de $\langle S \rangle$
	lineal de finitos de ellos.
	\section{Sistemas de Ecuaciones}
	\subsection{Generadores a Ecuación}
	Hallar una sistema de ecuaciones para el sub. generado por $\langle v_i\rangle$.

	Tenemos que son exactamente los $X$ tal que el sist $
	(v_i \mid X)
	$ tiene solución, entonces triangulamos y le pedimos que sea compatible.
	\subsection{Ecuación a Generadores}
	Tomamos el homogéneo y lo resolvemos

	\section{Dependencia Lineal}
	Un conjunto $X$ es ld si existe alguna combinación lineal no trivial de vectores en
	$X$ que de $0$

	\section{Ejemplo}
	Dado $\R^\N$ y la base canónica $e_i \; n = n == i$, tenemos que
	$\{e_i\} \cap \text{ const } 1$ es li

	Supongamos que no, luego $\sum \lambda_i x_i = 0$, si todos los $x_i$ son de la
	canónica, ya estamos
	Si no, wlog $x_1 = \text{ const } 1$, y tenemos que a partir de un momento
	$\left(\sum \lambda_i x_i\right) n$
	es siempre $\lambda_1$ para $n$ suficientemente grande,
	luego $\lambda_1 = 0$ y
	volvemos al caso anterior

	\section{Bases}
	Dado un $\V$ y un subconjunto $S \subseteq \V$ finito, decimos que $S$ es base iff
	$\langle S \rangle = \V$ y $S$ es li

	Decimos que $\dim \V = |S|$

	\subsection{Prop}
	si $\dim \V = n$ y $|S| = n$, tenemos:
	
	$S$ base iff $\langle S \rangle = \V$ iff $S$ li

	\subsection{Extender a una Base}
	Si tenemos un conjunto de vectores $S \subseteq \R^n$, luego triangulamos la
	matriz \(
	\begin{bmatrix}
		v_i \\
		\vdots \\
		e_i
	\end{bmatrix}
	\)

	\subsection{Extender a Bases de otros Conjuntos}
	Extender $\{(1 \; -1 \; 0 \; 1) \; (0 \; 1\; 0 \; -1)\}$ a una base de
	$T = \{x \in \R^4 \mid x_2 + x_4 = 0\}$.

	Notemos que $(0 \; 0\; 1\;0)$ está en $T$ y es li con los que tenemos, entonces
	extenderla con ele alcanza

	\subsection{Ejercicio 3}
	Hallar un conjunto de generadores finit y una base de $S = \{P \in \R[x]_{\leq 5} \mid
		P(0) = P(2) = 0, P(1) = P(-1)
	\}$

	Tenemos $P(x) = x(x-2)(a_0 + a_1x + a_2x^2 + a_3x^3)$

	$P(1) = -a_0 - a_1 - a_2 - a_3$

	$P(-1) = 3a_0 - 3a_1 + 3a_2 - 3a_3$

	Luego $-2 a_0 + a_1 - 2_a2 + a_3 = 0$

	$a_1 = 2a_0 + 2a_2 - a_3$ 

	Y a partir de eso armamos los generadores

	\section{Intersección Eq-Gen}
	Si se tiene un sub. expresado por generadores y uno expresado por ecuaciones, luego
	expresamos el de generadores de forma paramétrica y lo enchufamos en las ecuaciones
\end{document}
