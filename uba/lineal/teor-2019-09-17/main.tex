\documentclass{article}
\usepackage{amssymb}
\usepackage{mathtools}

\everymath{\displaystyle}
\setlength{\parskip}{3mm}
\setlength{\parindent}{0mm}

\def\R{\mathbb{R}}
\def\C{\mathbb{C}}
\def\N{\mathbb{N}}
\def\Q{\mathbb{Q}}

\date{}
\author{}

\begin{document}
\section*{Anulador}
Dado $S$ un subespacio de $\V$, si definimos
$T = \{x : \V : \forall \phi \in S^0 : \phi x = 0\}$, luego $S = T$.

Es claro que $S \inn T$. Vemos que $T \inn S$. Sea $x \in T$, $x \notin S$,
tomemos $v$ una base de $\V$, con $v_{i < r}$ base de $S$, y $v_r = x$. Sea
la base dual $\phi = v^*$. Como $\phi_{i \geq r}v_{i < r} = 0$, tenemos que
$\phi_{i \geq r}$ es base de $S^0$.

Luego $\phi_r \inn S^0$, entonces $\phi_r x = 0$, pero $x = v_r$, pero
entonces por definición de base dual tenemos $\phi_r x = 1$. Contradicción.

\subsection*{Observación}
Dado $\V$ finitamente generado y $S$ sub de $\V$, lugo si $\phi$ es base de
$S^0$, luego $S = \{x \in \V : \forall i : \phi_ix = 0\}$.

\section*{Doble dual}
Tenemos la función $\ev \in (\V \to \V^{**} = \V \to \V^{*} \to \K)$
dada por $\ev_x \phi = \phi x$. La función $\ev$ también se llama $\Phi$.

Esta función es $\iso$ para $\V$ finitamente generado.

\section*{Anulador del Anulador}
Tenemos $S^{00} = \{\Phi : \V^{**} : \forall \phi \in S^0 : \Phi\phi = 0\}$.
Por lo anterior, tenemos que
\[
S^{00} = \{\Phi \in \V^{**} : \forall \phi \in S^0 : \Phi\phi = 0\} =
\]
\[
\{\ev_x : x \in \V : \forall \phi \in S^0 : \ev_x\phi = 0\} =
\]
\[
\{\ev_x : x \in \V : \forall \phi \in S^0 : \phi x = 0\} =_{\iso}
\]
\[
\{x : x \in \V : \forall \phi \in S^0 : \phi x = 0\} = S
\]
Luego tenemos $S^{00} =_{\iso} S$

\section*{Operaciones de Subespacios sobre Anulación}
Se cumple que:
\begin{itemize}
	\item $(S+T)^0 = S^0 \cap T^0$
	\item $(S\cap T)^0 \inn^T S^0 + T^0$
\end{itemize}
Y para dimensión finita,
\begin{itemize}
	\item $(S+T)^0 = S^0 \cap T^0$
	\item $(S\cap T)^0 = S^0 + T^0$
\end{itemize}

\section*{Función Transpuesta}
Dada $f : \V \to \W$, la función transpuesta $f^t : \W^* \to \V^*$ es la dada
por $f^t \phi = \phi \circ f$, es decir $f^t = (\circ f)$. notemos que
$(*^t) : (\V \to \W) \to (\W^* \to \V^*)$ es lineal. Esto es ya que:
\begin{itemize}
	\item $f^t \phi$ es lineal.
	\item $f^t$ es lineal
	\item $(*^t)$ es lineal
\end{itemize}

\subsection*{Transpuesta de la Identidad}
Dada $\id : \V \to \V$, tenemos que $\id^t = \id_{\V^*}$

\subsection*{Composición}
Dada $f : \U \to \V$ y $g : \V \to \W$, tenemos que $(g \circ f)^t = f^t \circ
g^t : \W^* \to \U^*$.

Demo:
\[(g \circ f)^t \phi\]
\[= \phi \circ (g \circ f)\]
\[= (\phi \circ g) \circ f\]
\[= g^t \phi \circ f\]
\[= f^t (g^t \phi)\]
\[= (f^t \circ g^t) \phi\]

\subsection*{Isomorfismo}
Si $f : \V \to_{\iso} \W$, luego $f^t : \W \to_{\iso} \V$, y además $(f^{-1})^t
= (f^t)^{-1}$

Notemos que \[f^t \circ (f^{-1})^t = (f^{-1} \circ f)^t = \id^t = \id\]
y similarmente, \[(f^{-1})^t \circ f^t = (f \circ f^{-1})^t = \id^t = \id\]

\subsection*{Anuladores}
Si $f: \V \to \W$, luego $\ker f^t = (\Ima f)^0$. Además, $\Ima f^t \inn (\ker
f)^0$, la igualdad dándose para dimensión finita.

Notemos que
\[(\Ima f)^0 = \{\phi : \phi (\Ima f) = 0\} = \]
\[\{\phi : (\phi \circ f) \V = 0\}\]
\[\{\phi : f^t \phi \V = 0\}\]
\[\{\phi : f^t \phi = 0\} = \ker f^t\]

Supongamos $f^t \psi = \phi$, tenemos
\[f^t \psi = \phi\]
\[\psi \circ f = \phi\]
\[(\psi \circ f) \ker f = \phi \ker f\]
\[\psi (f \ker f) = \phi \ker f\]
\[\psi 0 = \phi \ker f\]
\[0 = \phi \ker f \To \phi \inn (\ker f) ^0\]
\end{document}
