\documentclass{article}
\usepackage{amssymb}
\usepackage{mathtools}

\everymath{\displaystyle}
\setlength{\parskip}{3mm}
\setlength{\parindent}{0mm}

\def\R{\mathbb{R}}
\def\C{\mathbb{C}}
\def\N{\mathbb{N}}
\def\Q{\mathbb{Q}}

\date{}
\author{}

\begin{document}
\section*{Jordan de Nilpotentes}
Dada una $f : \V \to \V$ nilpotente. de índice $k$, existe base $\BB$ de $\V$ tal que:
\[
	[f]_\BB =
	\begin{bmatrix}
		J_1 & 0 & 0 & 0 \\
		0 & J_2 & 0 & 0 \\
		0 & 0 & J_3 & 0 \\
		0 & 0 & 0 & J_4 \\
	\end{bmatrix}
\]

Demo:
Sabemos que
\[\{0\} = \ker f^0 \subsetneq \ker f^1 \subsetneq \dots \subsetneq \ker f^k = \V\]

Luego construyamos los $\CC_i$ en orden reverso. Tenemos que:
$\CC_{i}$ es una extensión $\CC_i$ de la base $f(\CC_{i+1})$ tal que $\ker f^{i-1} \oplus \bigoplus_{j \geq i} \langle \CC_j\rangle = \V$.

Después tomemos $\BB = \bigcup \CC_i$, que (reordenada) es la base que queremos.
\section*{Ejemplo 1}
Sea
\[
	A =
	\begin{bmatrix}
		0 & 0 & 0 & 0 & 0 & 0 \\
		1 & 0 & 0 & 0 & 0 & 0 \\
		-1 & -1 & 0 & 0 & 0 & 0 \\
		0 & 1 & 0 & 0 & 1 & 0 \\
		-1 & 0 & 0 & 0 & 0 & 0 \\
		1 & 0 & 0 & 0 & -1 & 0 \\
	\end{bmatrix}
\]
Tenemos
\[\ker A = \langle e_3, e_4, e_6 \rangle\]
\[\ker A^2 = \langle e_3, e_4, e_6, e_2, e_5 \rangle\]
Para construir la base de Jordan, tomemos:
Luego tomemos $C_3 = \{e_1\}$ alguna extensión de la base de $\ker A^2$ a una de $\ker A^3 = \V$.

Ahora queremos extender $\{f(e_1)\}$ a una base de $\ker A^2$, vemos que con $\{fe_1, e_5\}$ alcanza.

Ahora queremos extender $\{f^2e_1, fe_5\}$ a una base de $\ker A$, y con $\{f^2 e_1, fe_5, e_3\}$ alcanza.

Luego $\BB = \{e_1, fe_1, f^2e_1, e_5, fe_5, e_3\}$, y en esa base:
\[
	[A]_\BB =
	\begin{bmatrix}
		0 & 0 & 0 & 0 & 0 & 0 \\
		1 & 0 & 0 & 0 & 0 & 0 \\
		0 & 1 & 0 & 0 & 0 & 0 \\
		0 & 0 & 0 & 0 & 0 & 0 \\
		0 & 0 & 0 & 1 & 0 & 0 \\
		0 & 0 & 0 & 0 & 0 & 0 \\
	\end{bmatrix}
\]
Que tiene $3$ bloques de Jordan, de tamaños decrecientes.

\section*{Lemita}
Si $J \in \K^{n \times n}$ es un bloque de Jordan, luego $\rg J^i = n-i$.

\section*{Unicidad de la forma de Jordan}
Tenemos que:
\begin{enumerate}
	\item El tamaño del bloque de Jordan más grande es el índice de nilpotencia.
	\item La cantidad de bloques de tamaño mayor que $i$ que aparece en la forma de jordan es $\rg A^i - \rg A^{i+1}$
\end{enumerate}

Demo: sabemos que $m_A = x^k$, con indice de nilpotencia $k$. Además sabemos que $m_A = \lcm x^{b_i} = x^{\max b_i}$, donde $b_i$ son los tamaños de los bloques, luego $x^k = x^{\max b_i}$, luego está $1)$.

Lo segundo es trivial viendo como actúan las potencias de la forma de jordan de $A$.

Usando estas dos cosas, la forma de jordan de $A$ está unívocamente determinada por los números $\rg A^0, \rg A^1, \dots$, luego ya está.
\end{document}
