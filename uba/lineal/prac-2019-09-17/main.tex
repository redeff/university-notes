\documentclass{article}
\usepackage{amssymb}
\usepackage{mathtools}

\everymath{\displaystyle}
\setlength{\parskip}{3mm}
\setlength{\parindent}{0mm}

\def\R{\mathbb{R}}
\def\C{\mathbb{C}}
\def\N{\mathbb{N}}
\def\Q{\mathbb{Q}}

\date{}
\author{}

\begin{document}
\section*{Ejercicio 1}
Sea $S = \langle (1 \; 2 \; 3), (2 \; 2 \;  4) \rangle$. Halar ecuaciones que
describan $S$.

Tomamos la base $\BB = \{(1 \; 2 \; 3) , (2 \; 2\; 4) , (0 \; 0 \; 1)\}$ de
$\R^3$, y tomamos la base dual $\BB^* = \{\phi_1, \phi_2, \phi_3\}$. Luego
tenemos que $S^0 = \langle \phi_3 \rangle$, donde:

\[\phi_3 \; (1 \; 2 \; 3) = 0\]
\[\phi_3 \; (2 \; 2 \; 4) = 0\]
\[\phi_3 \; (0 \; 0 \; 1) = 1\]
Entonces ya estamos

\section*{Ejercicio 2}
Tomemos $
\BB = \left\{
	\begin{pmatrix}
		1 & 0 \\
		1 & 0
	\end{pmatrix}
	\begin{pmatrix}
		0 & 1 \\
		0 & 1
	\end{pmatrix}
	\begin{pmatrix}
		1 & 1 \\
		0 & 0
	\end{pmatrix}
	\begin{pmatrix}
		0 & 0 \\
		1 &-1 
	\end{pmatrix}
\right\}
$ base de $R^{2 \times 2}$.

Sea $\phi = \BB^*$. Sea $S \inn \V$ tal que $S^0 = \langle \phi_1 - \phi_3 +
\phi_4, \phi_2 - 2\phi_4 \rangle$. Hallar $S$.

Tenemos que $S = \{A : (\phi_1-\phi_3+\phi_4) A = 0 = (\phi_2 - 2\phi_4) A\}$.

Si tenemos que si $A_\BB = \alpha$, entonces $(\phi_1-\phi_3+\phi_4) A = 
\alpha_1 - \alpha_3 + \alpha_4
$, y $(\phi_2 - 2\phi_4) A = \alpha_2 - 2\alpha_4$. Entonces tenemos el sistema
de ecuaciones
\[
	\begin{cases}
		\alpha_1 - \alpha_3 + \alpha_4 &= 0 \\
		\alpha_2 - 2\alpha_4 &= 0
	\end{cases}
\]

Resolviendo nos queda $S = \left\langle
\begin{pmatrix}
	2 & 1 \\
	1 & 0
\end{pmatrix}
\begin{pmatrix}
	-1 & 2 \\
	0 & 1
\end{pmatrix}
\right\rangle$

\section*{Ejercicio 3}
Sea $\BB' = \{\phi_1, \phi_2, \phi_3\}$ base de $\R^{3*}$. Sean $v_1, v_2, v_3
\in \R^3$ tales que:

\begin{itemize}
	\item $\langle v_1, v_2 \rangle ^0 = \langle 2\phi_1-2\phi_2+3\phi_3 \rangle$
	\item $\langle v_3 \rangle ^0 = \langle 4\phi_1+2\phi_2+5\phi_3, \phi_1 +
		2\phi_2 + \phi_3 \rangle$
\end{itemize}

Notemos que \[n-\dim \langle v_1, v_2, v_3 \rangle = \dim \langle v_1, v_2,
v_3 \rangle^0 = \dim (\langle v_1, v_2 \rangle + \langle v_3 \rangle)^0 = \]
\[\dim(\langle v_1, v_2 \rangle + \langle v_3 \rangle)^0$
\[\dim\langle v_1, v_2 \rangle^0 \cap \langle v_3 \rangle^0$
\[\dim\langle 2\phi_1-2\phi_2+3\phi_3 \rangle \cap \langle 4\phi_1+2\phi_2
	+5\phi_3, \phi_1 +
2\phi_2 + \phi_3 \rangle\]
\[\dim\langle (2 \; -2 \; 3) \rangle \cap \langle (4 \; 2 \; 5) (1 \; 2 \; 1)
\rangle = 1\]

Luego el resultado es $2$

\section*{Ejercicio 4}
Sean $f,g,h \in \V^*$, demostrar que $T : \V \to \K^3, Tv = (fv, gv,
hv)$. Probar que $T$ es epi iff $\{f, g, h\} \li$.

Si es epi, tomemos $v_1, v_2, v_3$ que dan los canónicos evaluados en $T$,
luego si tomamos una cominación lineal $\alpha f + \beta g + \gamma h = 0$,
tenemos que $\alpha = 0$, ya que evaluamos en $v_1$ y así.

Si $\alpha f + \beta g + \gamma h = 0$, luego sabemos que $\alpha fv + \beta gv
+ \gamma hv = 0$. WLOG $\alpha \neq 0$ y tenemos que el $(1 \; 0 \; 0)$ no se
genera.

Tomemeos la transpuesta $T^t : (\K^3)^* \to \V^*$, donde $T^t \phi = \phi \circ
T$.

Tenemos que $\dim \Ima T^t$ es la dimansión del generado por la canónica, que
es $\langle f, g, h \rangle$

\section*{Ejercicio 5}
\end{document}
