\documentclass{article}
\usepackage{amssymb}
\usepackage{mathtools}

\everymath{\displaystyle}
\setlength{\parskip}{3mm}
\setlength{\parindent}{0mm}

\def\R{\mathbb{R}}
\def\C{\mathbb{C}}
\def\N{\mathbb{N}}
\def\Q{\mathbb{Q}}

\date{}
\author{}

\begin{document}
\section*{Determinantes}
\begin{itemize}
	\item $\det AB = \det A \cdot \det B$
	\item $\det A = \det A^t$
	\item $\det A \neq 0 \iff \rg A = n \iff A$ inversible
	\item $\forall i: \det A = \sum_j (-1)^{i+j} A_{ij} \cdot \det \; A(i|j)$
	\item $\forall j: \det A = \sum_i (-1)^{i+j} A_{ij} \cdot \det \; A(i|j)$
\end{itemize}

\section*{Ejercicio 1}
Tenemos:
\[
	\det
	\begin{bmatrix}
		a & b & c \\
		d & e & f \\
		g & h & i
	\end{bmatrix}
	= -1
\]
Calcular el determinante de:
\[
	X = 
	\begin{bmatrix}
		3b+3e+3h & 3c+3f+3i & 3a + 3d + 3g \\
		e-2h & f - 2i & d - 2g \\
		b + 2e -2h & c+2f-2i & a+2d-2g
	\end{bmatrix}
\]

\[
	\det \begin{bmatrix}
		3b+3e+3h & 3c+3f+3i & 3a + 3d + 3g \\
		e-2h & f - 2i & d - 2g \\
		b + 2e -2h & c+2f-2i & a+2d-2g
	\end{bmatrix} = 
\]
\[
	\det \begin{bmatrix}
		3a + 3d + 3g & 3b+3e+3h & 3c+3f+3i\\
		d-2g & e-2h & f - 2i \\
		a + 2d - 2g & b + 2e -2h & c+2f-2i
	\end{bmatrix} = 
\]
\[
	\det \begin{bmatrix}
		3 & 3 & 3 \\
		0 & 1 & -2 \\
		1 & 2 & -2
	\end{bmatrix}
	\begin{bmatrix}
		a & b & c \\
		d & e & f \\
		g & h & i
	\end{bmatrix}
\]
\[
	\det \begin{bmatrix}
		3 & 3 & 3 \\
		0 & 1 & -2 \\
		1 & 2 & -2
	\end{bmatrix} =
\]
\[
	\det \begin{bmatrix}
		0 & -3 & 9 \\
		0 & 1 & -2 \\
		1 & 2 & -2
	\end{bmatrix} =
\]

\section*{Ejercicio 2}
Dada la matriz en bloques:

\[
	X = 
	\begin{bmatrix}
		A & C \\
		0 & B
	\end{bmatrix}
\]
Demostrar $\det X = \det A \cdot \det B$
por inducción, sea $A \in \K^{n \times n}$:

\[\det X = \sum_{i \leq n} a_{i1} \det \; A(i|1) \cdot \det B = \det A \cdot \det B\]

Alternativamente, demostramos que $X \mapsto \det
\begin{bmatrix}
	X & C\\
	0 & B
\end{bmatrix}
$ es multilineal y estamos.

\section*{Matriz Compañera}
Tenemos la matriz:
\[
	A_n = 
	\begin{bmatrix}
		x & 0 & 0 & \dots &0& 0& a_1 \\
		-1 & x & 0 & \dots &0& 0& a_2 \\
		0 & -1 & x & \dots &0&0& a_3 \\
		\vdots & \vdots & \vdots & \ddots &\vdots& \vdots& \vdots \\
		0 & 0 & 0 & \dots &-1& x & a_{n-1} \\
		0 & 0 & 0 & \dots &0& -1 & x+a_n
	\end{bmatrix}
\]
Expandiendo por la primer fila, tenemos:
\[
	\det
	\begin{bmatrix}
		x & 0 & 0 & \dots &0& 0& a_1 \\
		-1 & x & 0 & \dots &0& 0& a_2 \\
		0 & -1 & x & \dots &0&0& a_3 \\
		\vdots & \vdots & \vdots & \ddots &\vdots& \vdots& \vdots \\
		0 & 0 & 0 & \dots &-1& x & a_{n-1} \\
		0 & 0 & 0 & \dots &0& -1 & x+a_n
	\end{bmatrix} = 
\]
\[
	x\det
	\begin{bmatrix}
		0 & 0 & \dots &0& 0& a_1 \\
		x & 0 & \dots &0& 0& a_2 \\
		-1 & x & \dots &0&0& a_3 \\
		\vdots & \vdots & \ddots &\vdots& \vdots& \vdots \\
		0 & 0 & \dots &-1& x & a_{n-1} \\
		0 & 0 & \dots &0& -1 & x+a_n
	\end{bmatrix} + 
	a_1 \det\begin{bmatrix}
		x & 0 & 0 & \dots &0& 0\\
		-1 & x & 0 & \dots &0& 0 \\
		0 & -1 & x & \dots &0&0 \\
		\vdots & \vdots & \vdots & \ddots &\vdots& \vdots \\
		0 & 0 & 0 & \dots &-1& x  \\
		0 & 0 & 0 & \dots &0& -1 
	\end{bmatrix}
\]
Que por inducción anda

\section*{Regla de Kramer}
Si tenemo un sistema $Ax = b$, entonces $x_i = \frac{\det \; B_i}{\det A}$, donde $B_i$ es $A$ pero con la $i$-ésima columba reemplazada por $b$.
\end{document}
