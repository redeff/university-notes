\documentclass{article}
\usepackage{amssymb}
\usepackage{mathtools}

\everymath{\displaystyle}
\setlength{\parskip}{3mm}
\setlength{\parindent}{0mm}

\def\R{\mathbb{R}}
\def\C{\mathbb{C}}
\def\N{\mathbb{N}}
\def\Q{\mathbb{Q}}

\date{}
\author{}

\begin{document}
\section*{Determinentes de la Adjunta}
Se tiene $\det \adj A = (\det A)^{n-1}$
\section*{Rango de la Adjunta}
Si $\rg A = n$, luego claramente $\rg \adj A = n$.

Si $\rg A \leq n-2$, luego toda submatriz de tamaño $n-1$ tiene determinante $0$, luego $\adj A = 0$.

Si $\rg A = n-1$, tenemos que $A \cdot \adj A = \det A \cdot I = 0$, luego tenemos que $\rg \adj A \leq 1$, ya que pensamos el producto como composición. Pero $\adj A \neq 0$, luego $\rg A = 1$.

\section*{Ejercicio}
Dadas:
\[
	A = 
	\begin{bmatrix}
		-6 & -1 & 11 \\
		10 & 0 & -15 \\
		1 & 1 & -1
	\end{bmatrix}
\]
\[
	B = 
	\begin{bmatrix}
		2 & 1 & 2 \\
		1 & 4 & 2 \\
		1 & 1 & 1
	\end{bmatrix}
\]

encontrar $\adj^{-1} A$, y demostrar que no existe para $B$.
Notemos que:
\[
	C \cdot \adj C = \Id \cdot \det C
\]
\[
	C \cdot A = \Id \cdot \det C
\]
\[
	C = A^{-1} \cdot \det C
\]
\[
	C = A^{-1} \cdot (\det A)^{\frac{1}{n-1}}
\]
\[
	C = A^{-1} \cdot (\det A)^{\frac{1}{2}}
\]
Para $A$ esto anda ya que $\det A = 25$. Pero para $\det B = -1$ no podemos, ya que ningún número al cuadrado da $-1$.

\section*{Ejercicio 2}
Probar que si $A \in \K^{n \times n}$ es triengular inferior, luego $\adj A$ también los es.

Tenemos que probar que si $i < j$, entonces $\det A(j|i) = 0$. Vamos a hacer eso probando que es triangular superior y le faltan cosas en la diagonal (Es fácil, hay que hacer los casitos)

\section*{Ejercicio 3}
Sea $A \in GL(n, \K)$ tq $\{A, \dots, A^n\}$ son li. Demostrar que $\{\adj A, \adj A^2, \dots, \adj A^n\}$ son li.

Si tomamos:
\[
	\sum \lambda_i \adj A^i
\]
\[
	A^n \sum \lambda_i \adj A^i
\]
\[
	\sum A^n \lambda_i \adj A^i
\]
\[
	\sum A^{n-i} A^i \lambda_i \adj A^i
\]
\[
	\sum \lambda_i A^{n-i} \det A^i
\]
Ya estamos

\section*{Comentario}
Si tenemos :
\[
	M = 
	\begin{bmatrix}
		A & B \\
		C & D
	\end{bmatrix}
	M^{-1} = 
	\begin{bmatrix}
		P & Q \\
		R & S
	\end{bmatrix}
\]
\end{document}
