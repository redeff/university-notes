\documentclass{article}
\input{../../../style/header.tex}
\begin{document}
\section*{Clase Experimental}
Dada $v$ base de $\V$, con $\phi = v^*$, se tiene las identidades:
\begin{itemize}
	\item $\forall x \in \V : ((x)_v)_i = \phi_ix$
	\item $\forall \phi \in \V^* : ((f)_\phi)_i = f v_i$
\end{itemize}

Si introducimos otras bases $u^* = \psi$, luego tenemos:

\[
(C_{uv})_{ij} = ((v_j)_u)_i = \psi_i v_j
\]
Además
\[
(C_{\phi \psi})_{ij} = ((\psi_j)_\phi)_i = \psi_j v_i
\]

\section*{Ejercicio 1}
Tomamos $\V = \R^3$, luego dados

$\phi_1 (x,y,z) = x$,

$\phi_2 (x, y, z) = y-z$, y

$\phi_3 (x,y,z) = 2x-y$

Decidir si son base.

Tomamos $(\phi_i)_{E^*}$ y vemos que son $\li$.

\section*{Ejercicio 2}
Sean $\V$ con $\dim V = n$, y $\phi \in \V^*, \phi \neq 0$, y sea $v \notin
\ker \phi$. Hay que demostrar que $\V = \langle v \rangle \oplus \ker \phi$.

Notemos que $\dim \Ima \phi = 1$, luego por teo de dimensión $\dim \ker \phi =
n-1$.

Además notemos que $\langle v \rangle \cap \phi = \{0\}$ por hipótesis, osea
que por teo de la dimensión ya estamos.

Osinó podemos escribir, dado $x \in \V$, que $x = (x - \lambda \cdot v) +
\lambda \cdot v$, donde queremos que $\phi (x - \lambda \cdot v) = 0 = \phi x
-\lambda \phi v$, es decir, alcanza con $\lambda = \frac{\phi x}{\phi v}$.

\section*{Ejercicio 5}
Dados $\phi, \psi \in \V^* - \{0\}$, probar que $\ker \phi = \ker \psi \iff$
\{\phi, \psi\}.

\subsection*{$\From$)}
Es trivial
\subsection*{$\To$)}
Supongamos $\ker \phi = \ker \psi$. Tomemos $v \notin \ker$, luego dijimos que
podemos escribir $\V = \langle v \rangle + \ker$, es decir, para todo $x$, se
cumple $x = \lambda v + z$. Luego notemos que $\phi x = \lambda \cdot \phi v$, y
$\psi x = \lambda \cdot \psi v$, luego $\phi x = \psi x \cdot \frac{\psi
v}{\phi v} $. Entonces ya estamos.

\subsection*{Ejercicio 6}
Tomemos $S = \{x \in \R^4 : 2x_1 + x_2 - x_3+ 3x_4 = 0, x_1-x_2+x_4 = 0\}$.

Notemos que $\phi_1 = 2x_1 + x_2 - x_3+ 3x_4$, $\phi_2 = x_1-x_2+x_4$, entonces
estos están en el generador. Además son li y da la dimensión. Entonces ya está.
\end{document}
