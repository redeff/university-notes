\documentclass{article}
\usepackage{amssymb}
\usepackage{mathtools}

\everymath{\displaystyle}
\setlength{\parskip}{3mm}
\setlength{\parindent}{0mm}

\def\R{\mathbb{R}}
\def\C{\mathbb{C}}
\def\N{\mathbb{N}}
\def\Q{\mathbb{Q}}

\date{}
\author{}


\begin{document}
\section*{Expresar en Base}
Dado $v \in \V$, y una base $\BB_\V$, tenemos que $(v)_{\BB_\V} = (\lambda_i)
\iff \sum \BB_{\V_i} \cdot \lambda_i = v$

\section*{Igualdad}
Si tenemos $f, g \in \Hom \V \; \W$, luego $f = g \iff [f]_{\BB_\V\BB_\W} =
[g]_{\BB_\V\BB_\W}$

\section*{Composición}
Si $f : \U \to \V$ y $g : \V \to \W$, luego $[g \circ f] = [g] \times [f]$

\section*{Corolario}
Tenemos $\V$, $\W$ de dimensiones finitas respectivamente; y $f \in \Hom \V
\; \W$. Luego $f \iso \iff [f]_{\BB_\V\BB_\W}$ es inversible.

\section*{Cambio de Base}
Tomemos dos bases $\BB_\V$ y $\BB_\V'$ de $\V$, con $\dim \V = n$, y lo mismo
para $\W$ de dimensión $m$. Tenemos que:
\[[f]_{\BB'_\V\BB'_\W} = C(\BB_\W, \BB'_\W) \times [f]_{\BB_\V\BB_\W}
\times C(\BB_\V, \BB'_\V)^{-1}\]

Donde $C(\AA, \BB) = [\id]_{\AA\BB}$.

\section*{Definición}
Dados $A, B \in \K^{n \times n}$, decimos que $A, B$ son semejantes sii:
\[A \sim B \iff \exists C \in \GL\K^n : A = C \times B \times C^{-1}\]

\section*{Interpretación de Semejanza}
Decimos que $A \sim B \iff \exists f \in \Hom \K^n \; \K^n :\exists \BB, \BB'
: [f]_\BB = A, [f]_{\BB'} = B$.

Demo: Trivial por cambio de base con $\V = \W$.

\section*{Rango de una Matriz}
Si tenemos una matriz $A \in \K^{n \times m}$, y una $f$ tal que $[f] = A$,
luego decimos que: $\rg A = \rg [f] = \dim \Ima f$

\subsection*{Dimensión de Soluciones de un Homogéneo}
Si tenemos $A \in \K^{m \times n}$, luego tenemos $\dim \{x : Ax = 0\} = n -
\rg A$.

\section*{Sistemas no Homogéneos}
Si tenemos el sistema lineal $Ax = b$, entonces tiene solución sii $\rg A = \rg
(A \mid b)$.

Notemos que $\rg A = \rg (A \mid b) \iff \dim \Ima A = \dim \Ima
(A | b) \iff \dim \langle A_{*i} \rangle = \dim (\langle A_{*i}
\rangle + \langle b \rangle) \iff$ b es combinación lineal de las columnas de
$A \iff Ax = b$ tiene solución.

\section*{Matrices Inversibles}
Demostremos que dada $A \in \K^{n \times n}$, luego $A \times B = I \To B
\times A = I$.

Tomemos $f : \K^{n \times n} \to \K^{n \times n}, fX = AX$. Luego notemos que
$\epi f$, ya que $f(B \times C) = A \times (B \times C) = (A \times B) \times C
= C$.

Luego por dimensión, tenemos $\iso f$. Pero además $f(B \times A - I) = A \times (B
\times A - I) = A \times B \times A - A = I \times A - A = 0$, luego $B \times
A = I$

\section*{Sistemas de Ecuaciones}
\begin{enumerate}
	\item $A$ inversible
	\item $Ax = 0$ tiene solución única
	\item $Ax = b$ tiene única solución
	\item $Ax = b$ tiene al menos una solución
	\item $\rg A = n$
\end{enumerate}
\end{document}
