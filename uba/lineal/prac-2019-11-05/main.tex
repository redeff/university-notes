\documentclass{article}
\usepackage{amssymb}
\usepackage{mathtools}

\everymath{\displaystyle}
\setlength{\parskip}{3mm}
\setlength{\parindent}{0mm}

\def\R{\mathbb{R}}
\def\C{\mathbb{C}}
\def\N{\mathbb{N}}
\def\Q{\mathbb{Q}}

\date{}
\author{}

\begin{document}
\section*{Vectores Cíclicos}
Si tenemos $f : \V \to \V$, luego existe un $v \in \V$ tal que $m_{f,v} = m_f$.

Tomemos la factorización
\[m_f = \prod P_i^{r_i}\]
con $P_i$ distintos e irreducibles en $\K$.

Luego sea:
\[Q_i = \frac{m_f}{P_i^{r_i}}\]
Notemos que:
\[
	\exists v_i : m_{Q_i(f), v_i} = P_i^{r_i}
\]
Ya que
\[
	\forall v_i : m_{Q_i(f), v_i} \mid P_i^{r_i}
\]
Y además se alcanza porque osinó el minimal no era mínimo.
Luego tomemos
\[v = \sum v_i\]

\section*{Semejanza en $\C^{3 \times 3}$}
Si tenemos dos matrices $A, B \in \C^{3 \times 3}$, luego:
\[
	A,B \text{ semejantes} \iff \chi_A = \chi_B \text{ y } m_a = m_b
\]

\section*{Condiciones para que exista matriz}
Si tenemos una matriz $A$, luego
\begin{itemize}
	\item $\dim \ker A = \#$ bloques
	\item $\dim \ker A^{i+1} - \dim \ker A^i = \#$ bloques de tamaño $> i$
	\item $\rg A^{i+1} - 2\rgA^i + \rg A^{i-1} = \#$ bloques de tamaño $= i$
\end{itemize}
\end{document}
