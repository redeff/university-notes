\documentclass{article}
\usepackage{amssymb}
\usepackage{mathtools}

\everymath{\displaystyle}
\setlength{\parskip}{3mm}
\setlength{\parindent}{0mm}

\def\R{\mathbb{R}}
\def\C{\mathbb{C}}
\def\N{\mathbb{N}}
\def\Q{\mathbb{Q}}

\date{}
\author{}

\begin{document}
\section*{Unicidad y Existencia de la Determinante}
Vamos a demostrar que existe una única multilineal alternante $f : \K^{n \times n} \to \K$ con $f \Id = 1$.

\subsection*{Existencia}
Si $g : \K^{n \times n} \to \K$ es mult alt, $g \Id = 1$, luego 
\[f : \K^{(n+1) \times (n+1)} \to \K, fA = \sum (-1)^{i+1} a_{1i}g(A(1|i))\]
es mult alt con $f \Id = 1$.

\subsection*{Unicidad}
Supongamos que hay una única $g : \K^{n \times n} \to \K$ mult alt, con $g \Id = 1$, y alguna mult alt $f : \K^{(n+1) \times (n+1)} \to \K$. Luego:

\[fA =
	\sum a_{i1} f 
	\begin{bmatrix}
		0 & a_{12} & \dots & a_{1n+1} \\
		0 & a_{22} & \dots & a_{2n+1} \\
		\vdots & \vdots & \ddots & \vdots \\
		1 & a_{i2} & \dots & a_{in+1} \\
		\vdots & \vdots & \ddots & \vdots \\
		0 & a_{n+12} & \dots & a_{n+1n+1} \\
	\end{bmatrix} =\]\[ 
	\sum a_{i1} f 
	\begin{bmatrix}
		0 & a_{12} & \dots & a_{1n+1} \\
		0 & a_{22} & \dots & a_{2n+1} \\
		\vdots & \vdots & \ddots & \vdots \\
		1 & 0 & \dots & 0 \\
		\vdots & \vdots & \ddots & \vdots \\
		0 & a_{n+12} & \dots & a_{n+1n+1} \\
	\end{bmatrix} =  \]\[
	\sum (-1)^{i+1} a_{i1} f 
	\begin{bmatrix}
		a_{12} & \dots &0& \dots & a_{1n+1} \\
		\vdots & \ddots & \vdots & \ddots & \vdots \\
		0 & \dots & 1 & \dots& 0 \\
		\vdots & \ddots & \vdots & \ddots & \vdots \\
		a_{n+12} & \dots & 0 & \dots & a_{n+1n+1} \\
	\end{bmatrix} = 
\]
Lo que nos queda es una multilineal alternada, que en la identidad da $1$, entonces tiene que ser $g$.

\section*{Determinante}
La determinante $\det: \K^{n \times n} \to \K$ es la única multilineal que manda $\Id$ a $1$. Tenemos entonces:

\[
	\det A
	= \sum (-1)^{i+1} a_{1i} \cdot \det \;A(1|i)
	= \sum (-1)^{i+1} a_{i1} \cdot \det \;A(i|1)
\]

\subsection*{Props}
\begin{itemize}
 \item Dada $A \in \K^{n \times n}$, luego $\det A = \det A^t$. Notemos que esto implica que la definición de multilinealidad alternada es indistinto de si la hacemos en filas o columnas.
 \item Dada $A$ diagonal superior, luego $\det A = \prod a_{ii}$. Trivial por inducción.
 \item Las determinantes de las operaciones son:
	 \begin{itemize}
		 \item $\det P^{ij} = -1$.
		 \item $\det M^i \alpha = \alpha$
		 \item $\det T^{ij}\alpha = 1$
	 \end{itemize}
 \item $\det AB = \det A \cdot \det B$

	 Demo: definamos $f : \K^{n \times n} \to \K$, $f X = \det AX$. Veamos que $f$ es multalt:

	 Tenemos $\det A(X_i | \dots) = \det (AX_i | \dots)$. Luego como $x \mapsto Ax$ es lineal y $\det$ es multilineal, tenemos $f$ multilineal. 

	 Luego necesariamente $f B = f \Id \cdot \det B$, pero $f\Id = \det A$, luego estamos.
\end{itemize}

\section*{Matrices Inversibles}
Notemos que $A$ es inversible sii $\det A \neq 0$.

Si $\det A = 0$, luego $\det AB = \det A \cdot \det B = 0 \neq 1 = \det I$, luego no es inversible

Si $\det A \neq 0$, entonces no hay columnas ld, porque osinó podemos hacernos una columna $0$ y estamos.

\section*{Adjunta}
Dada $A \in \K^{n \times n}$, la adjunta de $A$ está dada por:
\[(\adj A)_{ij} = (-1)^{i+j} \cdot \det \; A(j|i)\]

La adjunta cumple que $A \cdot \adj A = I \cdot \det A$

Demo: Tenemos
\[(A \cdot \adj A)_{ij} = \sum A_{ik} \cdot (\adj A)_{kj} = \]
\[\sum (-1)^{j+k} \cdot A_{ik} \cdot \det \; A(j|k)\]
Para $i = j$ tenemos:
\[\sum (-1)^{i+k} \cdot A_{ik} \cdot \det \; A(i|k) = \det A\]
Para $i \neq j$ tenemos:
\[\sum_k (-1)^{j+k} \cdot A_{ik} \cdot \det \; A(j|k) = \det B\]
Donde $B$ es la matriz que resulta de:
\[
	B_{ab} = 
	\begin{cases}
		A_{ab} & \text{Si $a \neq j$} \\
		A_{ib} & \text{Si $a = j$} \\
	\end{cases}
\]
Que claramente tiene determinente $0$, estonces estamos.
\subsection*{Inversos en $\Z$}
Una matriz $A \in \Z^{n \times n}$ tiene inverso sii $\det A = \pm 1$.

Notemos que si $AB = I$, luego $\det A \cdot \det B = 1$, entonces ambos son o $1$ o $-1$ (ya que son enteros).

Además, si $\det A = \pm 1$, entonces existe $\frac{1}{\det A}$, entonces podemos tomarnos $B = \frac{1}{\det A} \cdot \adj A$ y estamos.

\begin{itemize}
	\item ¿Dónde usamos que $\K$ es cuerpo?
	\item Si tomamos $SL(n,\Z) = \{ A \in \Z^{n \times n} : \det A = 1 \}$, forma un grupo bajo producto
	\item Toda matriz $A \in SL(n, \Z)$ se expresa como producto de elementales
	\item No existe un $c$ tal que toda matriz en $SL(2, \Z)$ se pueda expresar como producto de a lo sumo $c$ elementales
	\item Existe un $c$ tal que toda matriz en $SL(n \geq 3, \Z)$ se pueda expresar como producto de a lo sumo $c$ elementales
\end{itemize}
\end{document}
