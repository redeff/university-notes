\documentclass{article}
\usepackage{amssymb}
\usepackage{mathtools}

\everymath{\displaystyle}
\setlength{\parskip}{3mm}
\setlength{\parindent}{0mm}

\def\R{\mathbb{R}}
\def\C{\mathbb{C}}
\def\N{\mathbb{N}}
\def\Q{\mathbb{Q}}

\date{}
\author{}

\begin{document}
\section*{Repaso de Endomorfismos}
\subsection*{Autoadjunta}
\begin{itemize}
    \item $f = f^*$
    \item $(f)_\BB = \overline {(f)_\BB^t}$
    \item Diagonalizable con base bon de autovectores reales
    \item $(f)_\BB = UDU^*$ para $U$ unitaria y $D$ diagonal.
\end{itemize}
\subsection*{Ortogonal / Unitaria}
$f$ unitaria sii:
\begin{itemize}
    \item $\forall v,w : \langle fv, fw \rangle = \langle v, w\rangle$
    \item $f(\BB)$ es bon para todo $\BB$ bon.
    \item $(f)_\BB^* = (f)_\BB^{-1}$
\end{itemize}
\section*{Ejercicio 1}
Sea $f : \C^2 \to \C^2$, $f(x,y) = (2y,x)$. Hallar si existe pi tal que $f$ sea autoadjunta.

\begin{demo}
    Sabemos que $f$ tiene que ser diagonalizable. Diagonalicémosla.
    Tenemos:
    \[
        [f]_E =
        \begin{bmatrix}
            0 & 4 \\
            1 & 0 \\
        \end{bmatrix}
    \]
    Tenemos $\chi_f(\l) = (\l-2)(\l+2)$, luego es diagonalizable con autovalores reales.

    Los autovectores son además $v_{2} = (2,1)$, y $v_{-2} = (2,-1)$.

    Luego queremos que la base que diagonaliza la matriz $\BB = {v_2, v_{-2}}$ sea ortonormal, luego:
    \[\langle v_2, v_{-2} \rangle = 0\]
    \[\langle v_2, v_2 \rangle = 1\]
    \[\langle v_{-2}, v_{-2} \rangle = 1\]
    Pero nosotros queremos cómo es el producto de los de la canónica, luego tenemos que escribir:
    \[(1 \; 0) = \frac{1}{4}(2 \; 1) + \frac{1}{4}(2 \; -1)\]
    \[(1 \; 0) = \frac{1}{2}(2 \; 1) - \frac{1}{2}(2 \; -1)\]
    Y tenemos que:
    \[\langle e_1, e_2\rangle = 0\]
    \[\langle e_1, e_1\rangle = \frac{1}{8}\]
    \[\langle e_2, e_2\rangle = \frac{1}{2}\]
\end{demo}

\section*{Ejercicio 2}
Sea
\[
    A =
    \begin{bmatrix}
        1 & 2 & -2 & -4 \\
        0 & 3 & -1 & -3 \\
        0 & 2 & 0 & -1 \\
        0 & 1 & -1 & -1 \\
    \end{bmatrix}
\]
Decidir si para algún producto interno es autoadjunta o normal.
Tomemos:
\[
    \det
    \begin{bmatrix}
        \l-1 & -2 & 2 & 4 \\
        0 & \l-3 & 1 & 3 \\
        0 & -2 & \l & 1 \\
        0 & -1 & 1 & \l+1 \\
    \end{bmatrix} =\]
\[
    (\l - 1)((\l - 3)(\l(\l-1) - 1) + 2(\l+1-3) -(1 - 3\l))
\]
\[
    (\l - 1)((\l - 3)(\l(\l-1) - 1) + 2(\l+1-3) -(1 - 3\l)) =
\]
\[
    (\l - 1)((\l - 3)(\l^2-\l - 1) + 2\l-4 -1 + 3\l)
\]
\[
    (\l - 1)((\l - 3)(\l^2-\l - 1) + 5\l -5)
\]
\[
    (\l - 1)(\l^3-\l^2 - \l -3\l^2+3\l + 3 + 5\l -5)
\]
\[
    (\l - 1)(\l^3-2\l^2 + \l -2) =
\]
\[
    (\l - 1)(\l - 2)(\l + i)(\l - i)
\]
Luego es normal y no autoadjunta.
\section*{Clasificación de Ortogonales}
\subsection*{con $\dim \V = 2$}
Tenemos todas transformaciones son de la pinta:
\[(f)_\BB =
    \begin{bmatrix}
        \cos \theta & -\sin \theta \\
        \sin \theta & \cos \theta \\
    \end{bmatrix} \; \text{ó}
\]
\[(f)_\BB =
    \begin{bmatrix}
        1 & 0
        0 & -1
    \end{bmatrix}
\]
Para alguna $\BB$ bon.
\subsection*{con $\dim \V = 3$}
Hay tres casos:
\begin{itemize}
    \item Si $\det f = 1$, luego:
        \[
            (f)_\BB = \begin{bmatrix}
                1& 0& 0 \\
                0& \cos \theta & -\sin \theta \\
                0& \sin \theta & \cos \theta \\
            \end{bmatrix}
        \]
    \item si $\det f = -1$ y $1$ es autovalor:
        \[
            (f)_\BB = \begin{bmatrix}
                1& 0& 0 \\
                0& 1 & 0\\
                0& 0 & -1\\
            \end{bmatrix}
        \]
    \item si $\det f = -1$ y $1$ no es autovalor:
        \[
            (f)_\BB = \begin{bmatrix}
                1& 0& 0 \\
                0& 1 & 0\\
                0& 0 & -1\\
            \end{bmatrix} \cdot
            \begin{bmatrix}
                1& 0& 0 \\
                0& \cos \theta & -\sin \theta \\
                0& \sin \theta & \cos \theta \\
            \end{bmatrix}
        \]
\end{itemize}
\section*{Ejercicio 3}
Determinar todas las $f:\R^3 \to \R^3$ ortogonales tales que:
\[
    f(1 \; -1 \; 0) = (1 \; -1 \; 0)
\]
\[
    f(3 \; 1 \; 1) = (-1 \; -3 \; -1)
\]
Primero bonificamos la base $\BB = \{(1 \; -1 \; 0), (3 \; 1 \; 1)\}$, y sabemos a dónde va a parar eso.

Luego extendemos con un tercer vector $v$ ortonormal. Sabemos que $f v$ es ortonormal a dos cosas, por lo que tiene dos posibilidades. Cada una nos da una transformación válida.
\end{document}
