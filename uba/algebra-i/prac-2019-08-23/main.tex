\documentclass{article}
\usepackage{amssymb}
\usepackage{mathtools}

\everymath{\displaystyle}
\setlength{\parskip}{3mm}
\setlength{\parindent}{0mm}

\def\R{\mathbb{R}}
\def\C{\mathbb{C}}
\def\N{\mathbb{N}}
\def\Q{\mathbb{Q}}

\date{}
\author{}

\title{Funciones}

\begin{document}
	\maketitle
	\section{Definición}
	Una función $f : A \to B$ es una relación $f \subseteq A \times B$, donde
	$A$ se llama dominio y $B$ codominio
	que cumple $\forall a \in A \; \exists ! b : (a, b) \in f$, que se nota $fa = b$.

	\section{Imagen}
	Dado $C \inn A$, definimos $f C := \{f c : c \in C\}$. Al conjunto $fA$ se lo llama
	imagen de $f$ ($\Im f$).

	\section{Preimagen}
	Dado $D \inn B$, definimos $f^{-1}D = \{a \in A : fa \in D \}$.

	\section{Restricción}
	Dado $C \inn A$, definimos $f |_C : C \to B = f \cap (C \times B)$.

	\section{Composición}
	Dadas $f : A \to B$ y $g : B \to C$, definimos $g \circ f : A \to C$ como
	$(g \circ f) x = g (fx)$.

	\section{Inyectividad}
	Si $\forall a, b \in \text{Dom} f :  fa = fb \Rightarrow a = b$.

	\subsection{Prop}
	$f : A \to B$ es inyectiva $\iff \exists g : B \to A : g \circ f = \text{id}_A$.

	\section{Sobreyectividad}
	$f$ es sobreyectiva si $f A = B$.

	\subsection{Prop}
	$f : A \to B$ es sobreyectida $\iff \exists g : B \to A : f \circ g = \text{id}_B$.

	\section{Power Set}
	Se Nota $\mathcal{P}(X) = \{Y : Y \inn X\}$.

	\section{Ejercicio 1}
	Tenemos $f : \PP (\N) \to \PP (\N) \times \PP (\N)$ definida por $f A =
	(A \cup \{3\}, A \triangle \{3\})$.

	Demostrar que es inyectiva.

	Definamos $g : \PP(\N) \times \PP(\N) \to \PP(\N)$ como $g (\_, x) = x \triangle \{3\}$.
	Notemos que $A \triangle \{3\}\triangle \{3\} = A$, luego $g \circ f = \text{Id}_A$,
	luego es inyectiva.

	Además $(\emptyset, \emptyset)$ no está en la imagen, porque $3$ siempre pertenece
	al primer elemento del par, por lo tanto no es sobreyectiva.
\end{document}
