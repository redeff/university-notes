\documentclass{article}
\usepackage{amssymb}
\usepackage{mathtools}

\everymath{\displaystyle}
\setlength{\parskip}{3mm}
\setlength{\parindent}{0mm}

\def\R{\mathbb{R}}
\def\C{\mathbb{C}}
\def\N{\mathbb{N}}
\def\Q{\mathbb{Q}}

\date{}
\author{}

\title{Teórica de Álgebra 2}
\begin{document}
	\maketitle
	\section{Par Ordenado}
	El par ordenado se nota $(a, b)$ y tiene la propiedad de que
	$(a, b) = (c, d) \iff a=c \text{ y } b = d$

	El par $(a, b)$ se define como $\{\{a\}, \{a, b\}\}$, que notemos que cumple la prop
	del párrafo anterior

	Notemos que si $a = b$ luego $|(a, b)| = 1$, osino $|(a, b)| = 2$

	Supongamos $\{\{a\}, \{a, b\}\} = \{\{c\}, \{c, d\}\}$, luego $a = b$ iff
	$c = d$ para que tengan la misma cantidad de elementos.

	Si son iguales notemos que tenemos $\{\{a\}\} = \{\{c\}\}$, entonces $a = c$, y
	ya estamos

	Si son $a$ y $b$ distintos, luego ambos pares ordenados tienen tamaño $2$.
	Además para que $\{\{a\}, \{a, b\}\} = \{\{c\}, \{c, d\}\}$, necesariamente
	$\{a, b\} = \{c, d\}$ y $\{a\} = \{c\}$, ya que para que dos sean iguales deben
	tener el mismo tamaño

	Entonces $a = c$, y $\{a, b\} = \{c, d\}$, entonces $b = d$ y estamos

	\section{Geometría Analítica}
	Establece una biyección puntos del plano $\iff$ pares de reales

	\section{Preducto Cartesiano}
	dados $A$ y $B$, el producto cartesiano $A \times B := \{(a, b) \mid a \in A,
	b \in B\}$

	\section{Relaciones}
	Una relación entre elementos de $A$ y elementos de $B$ es un subconjunto $R
	\subseteq A \times B$
	\subsection{Relación Discreta}
	Cuando $R = \emptyset$
	\subsection{Relación Indiscreta}
	Cuando $R = A \times B$

	\subsection{Relación de Orden ($A = B$)}
	Una relación $\leq \;\subseteq A^2$ es un orden (parcial) si cumple:

	$a \leq a$ (Reflexividad)

	$a \leq b$ y $b \leq c$ implica $a \leq c$ (Transitividad)

	$a \leq b$ y $b \leq a$ implica $a = b$ (Antisimetría)
\end{document}
