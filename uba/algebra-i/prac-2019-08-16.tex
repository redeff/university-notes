\documentclass{article}
\usepackage{amssymb}
\usepackage{mathtools}

\everymath{\displaystyle}
\setlength{\parskip}{3mm}
\setlength{\parindent}{0mm}

\def\R{\mathbb{R}}
\def\C{\mathbb{C}}
\def\N{\mathbb{N}}
\def\Q{\mathbb{Q}}

\date{}
\author{}

\title{Segunda Práctica de Álgebra}
\begin{document}
	\maketitle
	\section{Ejercicio 1}
	Demostrar $B = \left(A \cap B^c\right) \cup \left(A^c \cap B\right) \iff A = \emptyset$

	Ida: Sea $x \in B$, luego $x \in A \cap B^c$ ó $x \in A^c \cap B$

	Si $x \in A \cap B^c$, contradicción, ya que tenemos $x \in B$ y $x \in B^c$, entonces
	$x \in A^c \cap B$, con lo que $B \subseteq A^c$, es decir $A$ y $B$ disjuntos

	Reemplazamos en la original y tenemos $B = A \cup B$, osea $A \subseteq B$, pero
	$A \subseteq B^c$, luego $A = \emptyset$

	Ahora la vuelta: si $A = \emptyset$, reemplazamos $B = (\emptyset \cap B^c)
	\cup (\emptyset^c \cap B) = \emptyset \cup B = B$, luego se cumple

	\section{Ejercicio 2}
	Demostrar $A \cap C = \emptyset \rightarrow A \cap (B \bigtriangleup C) = A \cap B$.

	Supongamos $x \in A \cap (B \bigtriangleup C)$, luego $x \in A$ y $x \in B
	\bigtriangleup C$, pero $x \in A$ implica $x \notin C$, luego $x \in B$, entonces
	$x \in A \cap B$

	Si $x \in A \cap B$, tenemos que $x \notin C$, luego $x \in B \bigtriangleup C$,
	entonces como $x \in A$ tenemos $x \in A \cap (B \bigtriangleup C)$
\end{document}
