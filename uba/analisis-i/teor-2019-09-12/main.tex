\documentclass{article}
\usepackage{amssymb}
\usepackage{mathtools}

\everymath{\displaystyle}
\setlength{\parskip}{3mm}
\setlength{\parindent}{0mm}

\def\R{\mathbb{R}}
\def\C{\mathbb{C}}
\def\N{\mathbb{N}}
\def\Q{\mathbb{Q}}

\date{}
\author{}

\begin{document}
\section*{Lagrange}
Dada $f: [a, b] \to \R$ continua y derivable, tenemos que existe $c$ tal que:
\[
	\frac{fa - fb}{a-b}  = f'c
\]
\section*{Cauchy}
Dadas $f, g: [a,b] \to \R$, contínuas en el cerrado, derivables en el aiberto,
luego existe $c$ tal que:
\[
	\frac{fa-fb}{ga - gb} = \frac{f'c}{g'c} 
\]
Para evitar problemas con el $0$, esto lo escribo como:
\[
	(fa-fb) \cdot g'c = (ga - gb) \cdot f'c
\]

Demo: tomamos \[hx = 
	(fa-fb) \cdot gx - (ga - gb) \cdot fx
	\]una combinación lineal de $f$ y $g$. Notemos que \[
	h'x = (fa-fb) \cdot g'x - (ga - gb) \cdot f'x
\]
Entonces tenemos que $h$ es contínua y derivable.

Notemos que $ha = -fb \cdot ga + gb \cdot fa$, y $hb = gb \cdot fa - fb \cdot
ga$, luego $ha = hb$, entonces por \emph{Rolle} estamos.

\section{Taylor}
Dado $f: (a-\e, a+\e) \to \R$ de clase $f \in \CC^k$, decimos:
$P_{k-1}x = \sum_{i < k} \frac{f^{(i)}a}{i!}  \cdot (x-a)$, y nos queda
$fx = R_{ka}x + P_{k-1}x$, donde
$R_{ka}x = \frac{f^{(k)}c}{k!}(x-a)^k$ para algún $c \in (a, b)$.

Demo: notemos que $R_{ka}^{(i)}a$ es $0$ para todo $i < k$. Y si tomamos
$Gx = (x-a)^k$, luego también $G^{(i)}a = 0$ para todo $i < k$.

Además, notemos que $G^{(k)}a = k!$, y $R^{(k)}a = f'a$.

Tomemos $\frac{Rx}{Gx} = \frac{Rx - Ra}{Gx-Ga} = \frac{R'c_1}{G'c_1} $ para
algún $c_1$.

Similarmente $\frac{R'c_1}{G'c_1} = \frac{R'c_1 - R'a}{G'c_1-G'a} =
\frac{R''c_2}{G''c_2} $,

así siguiendo, tenemos que $\frac{Rx}{Gx} = \frac{R^{(k)}c_k}{G^{(k)}c_k}$ para
algún $c_k \in (a, x)$, y tenemos:

\[
	\frac{Rx}{Gx} = \frac{f^{(k)}c}{k!} \iff Rx = \frac{f^{(k)}c}{k!} \cdot
	(x-a)^k
\]
Para algún $c \in (x, a)$.

Esto implica $|Rx| \leq \frac{\min \{f^{(k)}c\}_{c \in (a,x)}}{k!} \cdot
(x-a)^k$.

Notemos que $\lim_{k \to \infty} \frac{(x-a)^k}{k!} = 0$, luego si podemos
acotar las derivadas de $f$ en $(a, x)$, tenemos que el límite de los
polinomios es la función en cada punto.

Por ejemplo, para la unción $\sin$, tenemos que el taylor cuadrático tiene $R_2
= \left|\cos c \cdot \frac{x^3}{3!}\right| \leq \frac{|x|^3}{3!}$, es decir,
$\sin x = x + \OO(x^3)$

\section*{Tangentes}
Cuando tomábamos la definición de derivada, teníamos que se definía por el
límite:

\[
	\lim_{x \to p} \left|\frac{fx-fp}{x-p} -f'p\right| = 0
\]

Que expresado en función a la recta tangente tenemos:
\[
	\lim_{x \to p} \left|\frac{fx-[fp + (x-p) \cdot f'p]}{x-p}\right| = 0
\]

Donde $[fp + (x-p) \cdot f'p]$ es la recta tangente.

Equivalentemente, $fx = fp + (x-p) \cdot f'p + \oo (x-p)$

\section*{En varias Dimensiones}
Tomemos una $f : A \to \R$, con $A$ un abierto de $\R^n$. Y sea $p \in A$.

Vamos a querer buscar un plano tangente a la gráfica de $f$ en $(p, fp)$.

El plano va a ser parametrizado por $l = (x \mapsto fp
+ f'p \cdot (x-p)) \in \R^n \to \R$, donde $(.)$ es el
producto escalar y $f'p \in \R^n$.
Entonces queremos que el error tienda a $0$, es decir:

\[
	\lim_{x \to p} \frac{fx-lx}{||x-p||} = 0 \iff
\]

\[
	\lim_{x \to p} \frac{fx-[fp + (x-p) \cdot f'p]}{||x-p||} = 0
\]

si tomamos una dirección específica $d \in R^n$, podemos tomar el límite en la
recta de la forma $x = p + \lambda \cdot d$
dirección, es decir:
\[
	\lim_{\lambda \to 0} \frac{f(p + \lambda \cdot d)-
	l(p + \lambda \cdot d)}{|\lambda| \cdot ||d||} = 0 \iff
\]

\[
	\lim_{\lambda \to 0} \frac{f(p + \lambda \cdot d)-
	[fp + (\lambda \cdot d) \cdot f'p]}{|\lambda| \cdot ||d||} = 0 \iff
\]

Si tomamos $d = (1, 0)$, tenemos la derivada parcial, que se nota
\[\frac{\partial f}{\partial x_i} (p) =
	\lim_{\lambda \to 0} \frac{f(p + \lambda \cdot e_i)-
	fp}{\lambda}
\]

\section*{Diferenciabilidad}
Dada una función $f : A \to \R$, y $p \in A$, se dice $f$ \emph{diferenciable}
en $p$ sii:
\begin{itemize}
	\item existen las derivadas parciales $\frac{\partial f}{\partial x_i}(p)$
	\item $
			\lim_{x \to p} \frac{fx-\left[fp + \left(\frac{\partial f}{\partial x_i}
			(p)\right)_i \cdot (x-p)\right]}{||x-p||} = 0
		$
\end{itemize}
Vimos que para que un plano tangente ande, necesariamente $f' p =
\left(\frac{\partial f}{\partial x_i} (p)\right)_i$.

\section*{Derivada Direccional}
Dado $d \in \R^n$, se define:

\[
	\frac{\partial f}{\partial d}(p) =
	\lim_{\lambda \to 0} \frac{f(p + \lambda \cdot d)-
	fp }{|\lambda| \cdot ||d||}
\]

\end{document}
