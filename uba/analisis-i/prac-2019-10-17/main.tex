\documentclass{article}
\usepackage{amssymb}
\usepackage{mathtools}

\everymath{\displaystyle}
\setlength{\parskip}{3mm}
\setlength{\parindent}{0mm}

\def\R{\mathbb{R}}
\def\C{\mathbb{C}}
\def\N{\mathbb{N}}
\def\Q{\mathbb{Q}}

\date{}
\author{}

\begin{document}
	\section*{Teorema de la Función Inversa}
	Si tenemos $f : (A \inn \R^n) \to \R^n$, con $A$ abierto y $f \in \CC^1(A)$, luego si $\D f(x_0)$ es inversible para algún $x_0 \in A$, entonces existen entornos $V$ y $W$ de $x_0$ y de $fx_0$ tal que $f : V \to W$ es inversible y su inversa es $\CC^1$. Además:
	\[
		\forall x \in V : \D f^{-1} (fx) = [\D f(x)]^{-1}
	\]

	\section*{Ejercicio 1}
	Sea $f : \R^2 \to \R^2$ una $\CC^1$ tal que:
	\[
		f(2,3) = (1,3)
	\]
	\[
		\D f(2,3) =
		\begin{bmatrix}
			1 & -1 \\ 0 & 1
		\end{bmatrix}
	\]
	Sean $g, h : \R^2 \to \R^2$ dadas por:
	\[
		g(x,y) = (x^2-xy, x+y)
	\]
	\[
		h = f \circ g
	\]
	\begin{enumerate}
		\item Decidir si $h$ es inversible en algún entorno de $(2,1)$
		\item Sean $\gamma : \R^2 \to \R$, $\gamma (x,y) = 3x^2+y$. Hallar el poly de Taylor orden 1 de $k = \gamma \circ h^{-1}$ en $(1,3)$
	\end{enumerate}
\end{document}
