\documentclass{article}
\usepackage{amssymb}
\usepackage{mathtools}

\everymath{\displaystyle}
\setlength{\parskip}{3mm}
\setlength{\parindent}{0mm}

\def\R{\mathbb{R}}
\def\C{\mathbb{C}}
\def\N{\mathbb{N}}
\def\Q{\mathbb{Q}}

\date{}
\author{}

\begin{document}
\section{Subsucesión}
Dada una sucesión $a : \N \to A$, y un mapeo de indices $f : \N \to \N$
estrictamente creciente,
decimos que $a \circ f$ es una subsucesión de $a$.

\section{Conjunto Acotado}
Un conjunto $A \inn \R^d$ es acotado si está contenido en alguna bola.

\section{Teorema de Bolzano-Weierstrass}
Para toda sucesión acotada $a : \N \to \R^d$ de puntos existe una subsucesión convergente.

En $\R$, es fácil ya que tiene una subsucesión monótona, que será también acotada.

En $\R^d$ separamos coordenada a coordenada,
y vamos tomando subsucesiones de subsucesiones,
hasta que converja en cada coordenada, entonces converje en $\R^d$.

\section {Sucesiones Asintóticas}
Dadas dos sucesiones $a, b : \N \to \R$, se dicen que dos sucesiones son asintóticas
($a \sim b$) si $\lim_{i \to \infty} \frac{ai}{bi} = 1$.

Tenemos por ejemplo $p_n \sim n \ln n$.

\section{Límites}
Dada una función definida en un intervalo punteado de $x_0$,

\[f : (x_0 - \e, x_0 + \e) - \{x_0\} \to \R\]

Se dice 

\[\lim_{x \to x_0} f x = \ell \iff \forall \e > 0 : \exists \delta > 0
: \forall x : ||x - x_0|| < \delta \Rightarrow ||\ell - f x|| < \e\]
\end{document}
