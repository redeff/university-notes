\documentclass{article}
\usepackage{amssymb}
\usepackage{mathtools}

\everymath{\displaystyle}
\setlength{\parskip}{3mm}
\setlength{\parindent}{0mm}

\def\R{\mathbb{R}}
\def\C{\mathbb{C}}
\def\N{\mathbb{N}}
\def\Q{\mathbb{Q}}

\date{}
\author{}

\begin{document}
\section{Continuidad}
Dada $f : A \to \R$, con $A \inn \R^n$, se dice que es contínua en un $a \in A$
sii $\lim_{x \to a} fx = fa$.

Si no existe el límite, se la llama \emph{discontinuidad escencial},
pero si el límite existe pero no es $fa$, se dice que es una
\emph{discontinuidad evitable}.

\section{Ejercicio 1}
Estudiar la continuidad de $
f(x,y) = \left\{
	\begin{matrix}
		\frac{ x^5 \cdot y}{x^4 + y^4} & \text{Si $(x,y) \neq (0, 0)$} \\
		\; \\
		0 & \text{Caso Contrario}
	\end{matrix}
\right.
$ en $\R^n$

Para todo $P \neq (0, 0)$, claramente $f$ es continua en $P$, ya que es razón
de dos contínuas.

Ahora analicemos el límite en $P = (0, 0)$.

Notemos que
$\left|\frac{x^5 \cdot y}{x^4 + y^4}\right|
\leq \frac{||P||^6}{x^4 + y^4} \leq
\frac{||P||^6}{max\{x^2, y^2\}^2} \leq
\frac{||P||^6}{\left(\frac{||P||^2}{2}\right)^2}
\leq 4||P||^2$
, entonces ya estamos.

Alternativamente, 
$\left|\frac{x^5 \cdot y}{x^4 + y^4}\right|
\leq \frac{|x|^4}{x^4 + y^4} \cdot (x \cdot y) \leq x \cdot y \leq ||P||^2$

\section{Ejercicio 2}
Hallar todos los $\alpha \in \R$ tales que la función
\[
	f(x, y) = \left\{
		\begin{matrix}
			\frac{\ln (x+1) \cdot x^\alpha}{ |y+3|x^2 + y^2}
			& \text{Fuera del origen} \\
			\; \\
			0 & \text{Otherwise}
		\end{matrix}
	\right.
\]
es contínua.

Si $x = 0$, tenemos que el límite vale $0$.

Si $x \neq 0$ podemos escribir
$
\frac{\ln (x+1) \cdot x^\alpha}{ |y+3|x^2 + y^2} = \frac{\ln (x+1)}{x} \cdot
\frac{x^{\alpha+1}}{|y+3|x^2 + y^2}
$

Como el $\lim_{x \to 0} \frac{\ln (x+1)}{x} = 1$, no nos importa ese término.

$
\frac{x^{\alpha+1}}{|y+3|x^2 + y^2} \leq
\frac{x^{\alpha+1}}{|y+3|x^2} \leq
\frac{x^{\alpha+1}}{x^2}
$ Para $y$ suficientemente chico.

Pero $
\frac{x^{\alpha+1}}{x^2} \leq
x^{\alpha - 1}
$. Si $\alpha - 1 > 0$, esto se puede acotar por $||P||^{\alpha-1}$, y ya
estamos.

Si $\alpha < 1$, luego tomemos la curva $y = 0$, y tendremos:
$
\frac{x^{\alpha+1}}{|y+3|x^2 + y^2} =
\frac{x^{\alpha+1}}{3x^2} =
\frac{1}{3} \cdot x^{\alpha-1}
$. Que si $\alpha - 1 < 0$, este límite da infinito.

Ahora veamos que si $\alpha = 1$, este límite da $\frac{1}{3}$, osea que tampoco es
contínua.
\end{document}
