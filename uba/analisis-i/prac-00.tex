\documentclass{article}
\usepackage{amsmath}
\usepackage{amssymb}
\usepackage{mathtools}

\everymath{\displaystyle}
\setlength{\parskip}{3mm}
\setlength{\parindent}{0mm}

\def\R{\mathbb{R}}
\def\C{\mathbb{C}}
\def\N{\mathbb{N}}
\def\Q{\mathbb{Q}}

\date{}
\author{}

\title{Práctica de Análisis}
\begin{document}
\maketitle
\section{Desigualdad triangular}
Dados $a$, $b$, tenemos $|a + b| \leq |a| + |b|$, esta implica \[
	||a| - |b|| \leq |a-b|
\]

Demo

\begin{align*}
	|a+b| &\leq |a| + |b| \\
	\iff |a+b|^2 &\leq (|a| + |b|)^2 \\
	\iff (a+b)^2 &\leq a^2 + 2|a||b| + b^2 \\
	\iff a^2 + 2ab + b^2 &\leq a^2 + 2|a||b| + b^2 \\
	\iff 2ab &\leq 2|a||b| \\
	\iff ab  &\leq |ab| \\
\end{align*}

Demo de la otra
\[
	|a| = |a-b+b| \leq |a-b| + |b|
\]
\[
	|a| \leq |a-b| + |b|
\]
\[
	|a| - |b| \leq |a-b|
\]
Análogamente, 
\[
	|b| - |a| \leq |b-a| = |a-b|
\]
Luego 
\[
	max \; \{|b| - |a| \quad |a| - |b|\} \leq |a-b|
\]
\[
	||a| - |b|| \leq |a-b|
\]

\section{Supremo}
Dado $A \subseteq \R$, un $c \in \R$ es cota superior $\iff \forall a \in A: a \leq c$
\subsection{Acotadez}
Un conjunto $A \subseteq \R$ es acotado superiormente $\iff$ admite cota superior

Es acotado si es acotado superior y inferior
\subsection{Supremo}
Menor de las cotas superiores
\subsection{Axioma del supremo}
Para todo $A \subseteq \R$, con $A \ne \emptyset$ acotado superiormente,
existe el supremo llamado $\sup A$

\section{Máximo}
El máximo de un conjunto $A \subseteq \R$ vale $\sup A$ si $\sup A \in A$
\section{Arquímedes}
el conjunto $\R$ es arquimediano. es decir, para todo $x>0$, $y\in \R$, existe un $n \in N$
tal que $nx > y$

Supongamos que no, luego tomemos el conjunto $A = \{xn \mid n \in N\}$, luego $y$ es cota
superior, entonces tiene supremo, luego existe $s = \sup A$ cota superior de $A$
tal que hay un $t \in A$
con $t + x > s$; pero $t + x \in A \;\forall t \in A$, luego contradicción, la que $s$
ya no es cota superior

\section{Ejercicio}
Dado $A = \left\{\frac{n+1}{n} \;\middle|\; n \in N \right\}$, encontrar $\sup A$, $\max A$, $\inf A$ y $\min A$, si es que existen

Notemos que la sucesión $a_n = \frac{n+1}{n}$ es decreciente, luego $a_1 = 2 = \max A = \sup A$

Ahora vamos a demostrar que $1$ es ínfimo. Es claro que $1 < a_i \; \forall i \in N$, ahora necesitamos demostrar que para todo $\varepsilon > 0 \; \exists a_i < 1 + \varepsilon$

Tenemos
\[\frac{n+1}{n} < 1 + \varepsilon\]
\[\frac{n+1}{n} - 1 < \varepsilon\]
\[\frac{1}{n} < \varepsilon\]
\[1 < n\varepsilon\]
Que siempre existe por arquímedes

\section{Ejercicio 2}
Calcular $\sup \left\{\frac{n^2}{5^n} \;\middle|\; n \in N\right\}$.
Veamos que $a_n = \frac{n^2}{5^n}$ es decreciente
\[\frac{n^2}{5^n} \ge \frac{(n+1)^2}{5^{n+1}}\]
\[5n^2 \ge (n+1)^2\]
Que es verdad. Y como decrece, el primer elemento es el supremo y el máximo

\section{Ejercicio 3}
Calcular $\inf \left\{n^2 - 13n + 40 \;\middle|\; n \in N\right\}$

El vértice de la parábola está en $\frac{13}{2}$, entonces el mínimo se da en $6$ ó $7$
\end{document}
