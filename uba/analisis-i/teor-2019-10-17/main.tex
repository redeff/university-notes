\documentclass{article}
\usepackage{amssymb}
\usepackage{mathtools}

\everymath{\displaystyle}
\setlength{\parskip}{3mm}
\setlength{\parindent}{0mm}

\def\R{\mathbb{R}}
\def\C{\mathbb{C}}
\def\N{\mathbb{N}}
\def\Q{\mathbb{Q}}

\date{}
\author{}

\begin{document}
\section*{Mímimos en bordes}
Si tenemos el conjunto $C = \{(x,y) : x^2 + y^2 \leq 1\}$ y una función $f : \R^2 \to \R$ con $(1, 0)$ mínimo local de $f | _C$, luego:

Tomemos la curva $\gamma (t) = (\cos t, \sin t)$. Tenemos que:
\[
	\D (f \circ \gamma) (p) = (\D f) (\gamma p) \cdot (\D \gamma) p
\]
Sabemos que $\D (f \circ \gamma) 0 = 0$, ya que $\gamma 0 = p$, que es mínimo local. Luego:
\[
	0 = (\D f)(\gamma 0) \cdot (\D \gamma) 0
\]
\[
	0 = (\D f)p \cdot (1 \; 0)^t
\]

\section*{Funciones inversibles en $\R \to \R$}
Si tenemos una función contínua y estrictamente creciente $f : [a,b] \to [c,d]$con $fa = c$, $fb = d$, luego existe una función inversa $f ^ {-1} : \R \to \R$ contínua y estrictamente creciente.

La demostración de que es inyectiva sale por crecimiento, y la sobreyectividad sale por bolzano.

El crecimiento de la inversa sale trivial por el crecimiento de $f$.

Para la contonuidad, notemos que para todo $\e > 0$, si tomamos el intervalo $K = (f(x-\e), f(x+\e))$, luego notemos que $\forall y \in K: |f^{-1}(y) - f^{-1}(fx)| \leq \e$, luego $f^{-1}$ es contínua en $fx$.

\section*{Derivada de la Inversa}
Si tenemos $f$ derivable en $x_0$, luego:
\[
	(f^{-1})'(fx) = \frac{1}{f'x}
\]
Es decir:
\[
	(f^{-1})' \; y = \frac{1}{f'(f^{-1}y)}
\]

\section*{Funciones específicas}
Si tenemos $f = \sin$, luego:
\[
	(\sin^{-1})' \; y = \frac{1}{\sin' (\sin^{-1} y)}
\]
\[
	(\sin^{-1})' \; y = \frac{1}{\cos (\sin^{-1} y)}
\]
\[
	(\sin^{-1})' \; y = \frac{1}{\sqrt{\cos^2 (\sin^{-1} y)}}
\]
\[
	(\sin^{-1})' \; y = \frac{1}{\sqrt{1 - y^2}}
\]
\[
	\arcsin'  = t \mapsto \frac{1}{\sqrt{1-t^2}}
\]
\end{document}
