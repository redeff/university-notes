\documentclass{article}
\usepackage{amssymb}
\usepackage{mathtools}

\everymath{\displaystyle}
\setlength{\parskip}{3mm}
\setlength{\parindent}{0mm}

\def\R{\mathbb{R}}
\def\C{\mathbb{C}}
\def\N{\mathbb{N}}
\def\Q{\mathbb{Q}}

\date{}
\author{}

\begin{document}
\section*{Notación $\OO$ y $o$}
Decimos que $f x = \OO(g x) \iff \exists c > 0 : fx < c \cdot
gx$.

\section*{Derivada en $\R$}
Se tenemos $f : \R \to \R$, y $p \in \R$. Decimos que
\[f' \; p = \alpha \iff \lim_{h \to 0} \frac{f \; p - (f \; (p + h) + h
\alpha)}{||h||} \]

Decimos que $f x = o(g x) \iff \lim \frac{f x}{g x} = 0$
\section*{Deferenciación Multivaraible}
\subsection*{Plano Tangente}
Un plano dado por $\pi q = fp + \alpha(p-q)_x + \beta(p-q)_y$ es tangente a la
función $f$ en el punto $p$ sii:

\[
	\lim_{q \to p} \frac{fq - \pi q}{||p-q||} = 0
\]

\subsection{Ejemplo}
Dada la función $f \; xy = x^2 + y^3$, notemos que 
\[f \; (1+h) \; (2+k) = 1 + 2h + h^2 + 2^2 + 12k + 6k^2 + k^3 =\]
\[9 + (2h + 12k) + (h^2 + 2^2 + 6k^2 + k^3)\]

\subsection{Diferencial de $f$}
Dada una función $f : \R^n \to R$, la diferencial de $f$ en el punto $p$,
notada $\nabla f p \in \R^n$ dada por:
\[
	\lim_{q \to p} \frac{fq - fp - (q-p) \cdot \nabla f p}{||p-q||} = 0
\]
Cuando existe, se dice diferenciable, y se cumple:
\[(\nabla f p)_i = \frac{\partial f}{\partial e_i} \; p\]
Y en general, diferenciabilidad implica la existencia de los límites
direccionales y se cumple:
\[\frac{\partial f}{\partial v} \; p = v \cdot \nabla f p\]
En particular, con $||v||$ constante, la derivada direccional se maximiza con
$v$ paralelo a $\nabla fp$.

\section*{Teorema}
Sea $f : (\Omega \in \R^n) \to \R$, con $p \in \Omega^\circ$. Si las derivadas
parciales $\frac{\partial f}{\partial e_i} $ existen y son contínuas en un
entorno de $p$, entonces $f$ es diferenciable en $p$.
\end{document}
