\documentclass{article}
\usepackage{amssymb}
\usepackage{mathtools}

\everymath{\displaystyle}
\setlength{\parskip}{3mm}
\setlength{\parindent}{0mm}

\def\R{\mathbb{R}}
\def\C{\mathbb{C}}
\def\N{\mathbb{N}}
\def\Q{\mathbb{Q}}

\date{}
\author{}

\begin{document}
    \section*{Cambio de Variables en $\R^n$}
    Si $T : \R^n \to \R^n$ inyectiva y $\CC^1$. Sea $A \inn \R^2$ acotado, y $f : T(A) \to \R^2$ integrable, Si $\D T$ es inversible en $A$, luego:
    \[
        \int_{T(A)} f = \int_{A} (f \circ T) \cdot |\det \D T|
    \]
    \section*{Coordenadas Polares}
    Se puede parametrizar $\R^2$ con una función $\rho : [0, 2\pi) \times [0, +\infty) \to \R^2$, tal que:
    \[
        \rho(\theta, r) = (r \cos \theta, r \sin \theta)
    \]
    Para integrar, vamos a querer el módulo del Jacobiano de esta función:
    \[
        |J\rho| = \left| \det
        \begin{bmatrix}
            \cos \theta & -r\sin \theta \\
            \sin \theta & r \cos \theta
        \end{bmatrix}\right| = |r| = r
    \]
    \section*{Coordenadas Cilíndricas}
    Tenemos
    \[
        \rho(\theta, r, z) = (r \cos \theta, r \sin \theta, z)
    \]
    y el Jacobiano sigue quedando $r$.
    \section*{Coordenadas Esféricas}
    Se parametriza $\R^3$ como
    \[\xi(\theta, \phi, r) = (r \sin \phi \cos \theta, r \sin \phi \sin \theta, r \cos \phi)\]
    Y el diferencial es:
    \[
        D \xi =
        \begin{bmatrix}
            - r \sin \phi \sin \theta & r \sin \phi \cos \theta & 0 \\
            r \cos \phi \cos \theta & r \cos \phi \sin \theta & -r\sin \phi \\
            \sin \phi \cos \theta & \sin \phi \sin \theta & \cos \phi \\
        \end{bmatrix}
    \]
    Cuya determinante es $r^2 \sin \phi$.
    \section*{Ejercicio}
    Calcular volumen de:
    \[W = \{(x,y,z) \in \R^3 : x^2 + y^2 + z^2 \leq 1, x^2 + y^2 \leq z^2, z \geq 0\}\]
    Queremos:
    \[
        \iiint\limits_W 1
    \]
    Haciendo coordenadas polares, tenemos:
    \begin{itemize}
        \item $0 \leq r \leq 1$
        \item $0 \leq \theta \leq 2\pi$
        \item $0 \leq \phi \leq \frac{\pi}{4}$
    \end{itemize}
    Luego tenemos:
    \[
        \int_0^1\int_0^{2\pi}\int_0^{\frac{\pi}{4}} r^2 \sin \phi\; d\phi d\theta dr =
    \]
    \[
        \left(\int_0^1 r^2 \; dr\right) \left(\int_0^{2\pi} \; d\theta\right) \left(\int_0^{\frac{\pi}{4}} \sin \phi \; d\phi\right) =
    \]
    \[
        \frac{1}{3} \cdot 2\pi \cdot \left(1 - \frac{1}{\sqrt 2}\right) =
    \]
    \[
        \frac{\pi\left(2 - \sqrt 2\right)}{3}
    \]
    En coordenadas polares queda:
    \[
        \int_0^{2\pi} \int_0^1 \int_0^{\min \{z, \sqrt{1 - z^2}\}} r \; drdzd\theta =
    \]
    \[
        2\pi\left(
            \int_0^{\frac{1}{\sqrt 2}} \int_0^z r \; drdz +
            \int_{\frac{1}{\sqrt 2}}^1 \int_0^{\sqrt{1 - z^2}} r \; drdz
        \right)
    \]
    Que sale.
    Osinó, tenemos que hacer otra cosa y al final la cuanta nos queda:
    \[
        \int r \sqrt{1 - r^2} \; dr
    \]
    Pongo $x = 1 - r^2, dx = -2r \; dr$,
    \[
        \int -\frac{\sqrt{x}}{2} \; dx
    \]
\end{document}
