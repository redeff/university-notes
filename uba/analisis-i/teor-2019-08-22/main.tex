\documentclass{article}
\usepackage{amssymb}
\usepackage{mathtools}

\everymath{\displaystyle}
\setlength{\parskip}{3mm}
\setlength{\parindent}{0mm}

\def\R{\mathbb{R}}
\def\C{\mathbb{C}}
\def\N{\mathbb{N}}
\def\Q{\mathbb{Q}}

\date{}
\author{}

\begin{document}
\section{Espacio}
Se trabaja en el espacio $\R^d$, que es el espacio euclídeo $d$-dimensional.

\subsection{Espacio Vectorial}
$\R^d$ tiene estructura de \emph{espacio vectorial} bajo $\R$

\subsection{Norma y Distancia}
La norma  está definida como $||P|| = \sqrt{\sum (P_i)^2}$.
Propiedades de la norma:
\begin{itemize}
	\item $||P|| \geq 0$, y da $0$ solo para el vector $0$.
	\item $||\lambda \cdot P|| = |\lambda|\cdot||P||$.
	\item $||P+Q|| \leq ||P|| + ||Q||$ (Desigualdad Triangular).
	\item $||P-Q|| \geq \big|||P|| - ||Q||\big|$ (Desigualdad Triangular$'$).
\end{itemize}

La distancia en $\R^d$ se define como $d(P, Q) = ||P-Q||$.

\section{Límite en $\R^d$}
Notemos que $\N \to \{d\} \to \R \sim \{d\} \to \N \to \R$.

Dada $a : \N \to \{d\} \to \R$, se define el límite como:

\[\lim_{n \to \infty} an = \ell \iff \forall \e > 0 \;
	\exists n_0 \in \N \;\forall
n > n_0 : ||\ell - an|| < \e\]

\subsection{Bolas en $\R^d$}
Se dice que $B(P, \e) = \{Q \in \R^d : ||P - Q|| < \e\}$ es una bola abierta.

\subsection{Teorema}
Dada $a : \N \to \{d\} \to \R$, tenemos
$\lim_{n \to \infty} an = \ell \iff \forall i \in \{d\}, \lim_{n \to \infty} ani = \ell i$

O equivalentemente $\lim a = \lim \circ \; \text{swap} \; a$

\subsubsection{Demo}
Supongamos $\lim a = \ell$, luego tenemos $\forall \e > 0 \; \exists n_0 \; \forall n > n_0 :
||an - \ell|| < \e$

Notemos que $||an - \ell|| < |(an - \ell) i|| \;\forall i \in \{d\}$, luego estamos.

Ahora hagamos el otro lado. Supongamos que $\lim \circ \;\text{swap}\; a = \ell$.
Luego tomémonos, dado un $\e > 0$, un $n_0$ tal que $|ani - \ell i| < \frac{\e}{\sqrt{d}}
\forall n > n_0, i \leq d$

Luego tenemos que, a apartir de $n_0$, $||an - \ell|| = \sqrt{\sum (ani - \ell i)^2}
< \sqrt{\sum \left(\frac{\e^2}{d} \right)} = \sqrt{\e^2} = \e$, entonces estamos.

\section{Clasificación de Conjuntos}
Dado un conjunto $A \subseteq \R^d$, un punto $x \in A$ es \emph{interior} a $A$ si
existe un $\e$ tal que $B(x, \e) \subseteq A$. Un punto es \emph{exterior} si es interior
el complemento. Un punto está en la \emph{frontera} de $A$, si no está ni en el interior
ni en el exterior de $A$.

Los interiores a $A$ se notan $A^\circ$, los exteriores se notan $A^{\text{ext}}$, y la
frontera se nota $\partial A$.

\subsection{Abierto}
$A$ es abierto si $A = A^\circ$. $A$ es cerrado si $\partial A \subseteq A$.

Alternativamente, $A$ es cerrado si para toda $a : \N \to A$, si existe $\lim a$, luego
$\lim a \in A$
\end{document}
