\documentclass{article}
\usepackage{amssymb}
\usepackage{mathtools}

\everymath{\displaystyle}
\setlength{\parskip}{3mm}
\setlength{\parindent}{0mm}

\def\R{\mathbb{R}}
\def\C{\mathbb{C}}
\def\N{\mathbb{N}}
\def\Q{\mathbb{Q}}

\date{}
\author{}

\begin{document}
\section*{Taylor en una variable}
Si tomamos el polinomio $Px = \sum_{n \leq N} \frac{f^{(n)}a}{n!} \cdot (x - a)$, con $f \in \CC^N$, luego tenemos, por L'Hôpital repetido que:
\[
	\lim \frac{fx - Px}{x-a} = 0
\]

Además, por Lagrange, tenemos que existe un $c$ entre $a$ y $x$ con:
\[
	fx - Px = \frac{f^{(n+1)}c}{(n+1)!} (x-a)^{n+1}
\]

\section*{Matriz Hessiana}
Dada una $g : (A \inn \R^2) \to \R$, y un $p \inn A$, la matriz hessiana en $p$ de $g$ está dada pos:
\[
	\Hs g p = 
	\begin{bmatrix}
		g_{xx} \;p & g_{yx} \;p \\
		g_{xy} \;p & g_{yy} \;p
	\end{bmatrix} = 
	\begin{bmatrix}
		\nabla g_x \;p \\
		\nabla g_y \; p
	\end{bmatrix}
\]

En general, en $\R^n$, la matriz está dada por:
\[
	\Hs gp = \D^2 gp = 
	\begin{bmatrix}
		\vdots \\
		\nabla g_{x_i} \; p \\
		\vdots
	\end{bmatrix}
\]
Alternativamente, tenemos $\Hs g = \D^2 g$

Notemos que si $g \in \CC^2$, luego $\D^2 g p$ es simétrica.

\section*{Taylor Multivariable}
Dada $f : (A \inn \R^n) \to \R$, con $f \in \C^3$, con $A$ abierto convexo, luego tenemos el polinomio de taylor de orden dos:
\[
	fx = fp + \sum f_{x_i}p (x_i-p_i) + \frac{1}{2}\sum \sum f_{x_ix_j} (x_i-p_i)(x_j-p_j) + R(x)
\]

Donde $R(x) = \frac{1}{6} \sum \sum \sum f_{x_ix_jx_z} (x_i-p_i) (x_j - p_j)(x_k-p_k)$, que implica que $\frac{R(x)}{||x-p||^2} \to 0$

Alternativamente, el polinomio de taylor se puede escribir como
\[
	fx = fp + \nabla f \cdot (x-p) + \frac{1}{2} \Hs fp \cdot (x-p) \cdot (x-p) + R(x)
\]

Además, tenemos que el resto se escribe:
\[R(x) = \frac{1}{6} \sum \sum \sum f_{x_ix_jx_z} (x_i-p_i) (x_j - p_j)(x_k-p_k)\]
Por Lagrange en la composición de $f$ con la curva que mappea $[0,1]$ al segmento $\overline{xp}$

Además, como $f \in \CC^3$, en cualquier entorno compacto convexo de $P$, todas las derivadas terceras están acotadas, luego su suma con módulo está acotado por, digamos, $M$. Pero luego $R(x) \leq M \cdot ||x-p||^3$. Luego estamos.
\end{document}
