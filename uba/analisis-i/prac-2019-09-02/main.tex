\documentclass{article}
\usepackage{amssymb}
\usepackage{mathtools}

\everymath{\displaystyle}
\setlength{\parskip}{3mm}
\setlength{\parindent}{0mm}

\def\R{\mathbb{R}}
\def\C{\mathbb{C}}
\def\N{\mathbb{N}}
\def\Q{\mathbb{Q}}

\date{}
\author{}

\begin{document}
\section{Límite}
Dado una función $f : A \to \R$, con $A \inn \R$, luego para todo
$x_0 \in A$ decimos que:
\[\lim_{x \to x_0} fx = \ell \iff \forall \e>0 : \exists \delta > 0 :
\forall x \in A : |x - x_0| < \delta \To |fx - fx_0| < \e\]

\section{Ejemplos}
Tomemos $fx = x^2$, y demostremos $\lim_{x \to 2} fx = 4$.
tenemos $|x^2-4| = |x-2| \cdot |x+2|$. Por desigualdad triangular tenemos
$|x+2| < \delta + 4$, y tenemos $|x-2| < \delta$, luego tenemos:
$|x^2-4| < (\delta + 4) \cdot \delta$, entonces queremos un $\delta$ tal que
$(\delta + 4) \cdot \delta < \e$.

Si le pedimos que $\delta < 1$, tomemos $\delta < \min \left\{\frac{\e}{1 + 4}, 1 \right\}< \frac{\e}{\delta + 4}$, y ya estamos.

Alternativamente, usamos que $\delta^2 < \delta$ para $\delta < 1$, entonces tenemos
$(\delta + 4) \cdot \delta = \delta^2 + 4 \cdot \delta < 5 \cdot \delta$, luego
tomamos $\delta < \min \left\{\frac{\e}{5}, 1\right\}$ y estamos.

\section{Cero por Acotado}
Si tenemos $\lim_{x \to x_0} fx = 0$ y $g$ acotada, luego
$\lim_{x \to x_0} gx \cdot fx = 0$.

\section{L'Hôpital}
Si $f$ y $g$ con continuas y derivables en un entorno de $c \in \R$, y tenemos
$f c = g c = 0$ y $g' c \neq 0$. Luego si existe el límite
$\ell = \lim_{x \to c} \frac{f' x}{g' c}$, luego $\lim_{x \to c} \frac{f c}{g c} = \ell$.

\subsection{Def de Wikipedia}
Dadas funciones derivables $f$ y $g$ definidas en un entorno punteado $A \inn \R$
de $a \in \R$, si
$\lim_{x \to a} fx = \lim_{x \to a} gx = 0$ y $g x \neq 0 \; \forall x \in A$, y se tiene
$\ell = \lim_{x \to a} \frac{f' x }{g' x } $,
luego $\ell = \lim_{x \to a} \frac{fx }{gx } $

\section{Límite Multivariable}
Dado una función $f : A \to \R^m$, con $A \inn \R^n$, luego para todo
$x_0 \in \overline{A}$ decimos que:
\[\lim_{x \to x_0} fx = \ell \iff \forall \e>0 : \exists \delta > 0 :
\forall x \in A : ||x - x_0|| < \delta \To ||fx - fx_0|| < \e\]

\section{Ejemplos}
Demostremos $\lim_{(x, y) \to (-1, 3)} 2x + y = 1$. Tenemos que encontrar
$\delta$ tal que $|2x + y - 1| < \e$.

Tenemos $|x+1| < \delta$ y $|y-3| < \delta$.
$|2x+2| <2 \cdot  \delta$. Sumando las dos últimas por desigualdad triangular tenemos:

$|2x + y - 1| < 3 \cdot \delta$, entonces tomamos $\delta = \frac{e}{3}$, y ya estamos.
\end{document}
