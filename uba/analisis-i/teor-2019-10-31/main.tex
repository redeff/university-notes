\documentclass{article}
\usepackage{amssymb}
\usepackage{mathtools}

\everymath{\displaystyle}
\setlength{\parskip}{3mm}
\setlength{\parindent}{0mm}

\def\R{\mathbb{R}}
\def\C{\mathbb{C}}
\def\N{\mathbb{N}}
\def\Q{\mathbb{Q}}

\date{}
\author{}

\begin{document}
	\section*{Integral de Riemann}
	Dada una $f : [a,b] \to \R$, $f$ acotada.

	dada una partición $\pi = \{x_i\}_{i \leq n}$, con $x_0 = a, x_n = b$, definimos la suma inferior de riemann como:
	\[
		s(f, \pi) = \sum_{i = 1}^n |x_i - x_{i-1}| \cdot \inf \{f(x) : x_{i-1} \leq x \leq x_i \}
	\]
	Y la suma superior de riemann como:
	\[
		S(f, \pi) = \sum_{i = 1}^n |x_i - x_{i-1}| \cdot \sup \{f(x) : x_{i-1} \leq x \leq x_i \}
	\]
	Decimos que
	\[I = \sup_{\pi \text{ particion}} s(f, \pi)\]
	Que existe, ya que $s(f, \pi) \leq (b-a) \cdot \inf f [a,b]$. Similarmente:
	\[S = \inf_{\pi \text{ particion}} S(f, \pi)\]

	Se dice que $f$ es \emph{Riemann integrable} si $I = S$, y decimos que
	\[\int_a^b f(x) dx = I = S\]

	\section*{Sumas de Potencias}
	Ponele que queremos calcular
	\[S_n(T) \sum_i^T i^n\]
	Notemos que:
	\[
		\sum (i+1)^{n+1} - i^{n+1} = (i+1)^{n+1} - 1
	\]
	Por telescópica. Pero por otro lado:
	\[
		\sum (i+1)^{n+1} - i^{n+1} =
		\sum_i \left( \sum_j {n+1 \choose j} i^j \right) - i ^{n+1} =
	\]
	\[
		\sum_i^T \left( \sum_j^n {n+1 \choose j} i^j \right) =
	\]
	\[
		\sum_j^n {n+1 \choose j} \left( \sum_i^T i^j \right) =
	\]
	\[
		\sum_j^n {n+1 \choose j} S_j(T) = (i+1)^{n+1} - i^{n+1}
	\]
	\[
		S_n(T) = \frac{(i+1)^{n+1} - i^{n+1} - \sum_j^{n-1} {n+1 \choose j} S_j(T)}{n+1}
	\]

	\section*{Calculando Integrales}
	Si Tenemos una sucesión de particiones $(\pi_n)_n$, y tenemos que
	\[\lim_{n \to \infty} S(f, \pi) - s(f,\pi) = 0 \iff\]
	\[\lim_{n \to \infty} S(f, \pi) = \lim_{n \to \infty}s(f,\pi) = k\]
	Luego $\int (x) dx = e$

	Por ejemplo, si $f(x) = x$, y tomamos $(\pi_n)_i = \frac{i}{n}$, luego tenemos:
	\[
		S(f, \pi_n) = \sum_{i = 1}^n \frac{1}{n} \cdot \frac{i}{n} = \frac{1}{n^2} \cdot \frac{i(i+1)}{2}
	\]
	\[
		S(f, \pi_n) = \sum_{i = 1}^n \frac{1}{n} \cdot \frac{i-1}{n} = \frac{1}{n^2} \cdot \frac{i(i-1)}{2}
	\]
	Que ambas tienden a $\frac{1}{2}$ con $n \to \infty$.

	\section*{Función de Dirichlet}
	Definimos $f(x) = 
	\begin{cases}
		1 & \text{Si $x \in \Q$} \\
		0 & \text{Si $x \notin \Q$}
	\end{cases}$.

	Notemos que $S(f, \pi) = 1$ para todo $\pi$, y similarmente $s(f, \pi) = 0$, ya que $\Q$ y $\R - \Q$ son densos en $\R$.

	\section*{Integrabilidad de las Contínuas}
	Cualquier función contínua $f : [a,b] \to \R$ contínua es Riemann integrable.
	\section*{Teorema Fundamental del Cálculo}
	Dada $f : [a,b] \to \R$ contínua, si definimos $F(x) = \int_a^x f(t) dt$, luego tenemos $F$ continua en $[a,b]$ y derivable en $(a,b)$, y
	\[\forall x \in (a,b) : F'(x) = f(x)\]

	Asimismo, si tenemos $f : [a,b] \to_{\CC^1} \R$, luego
	\[\int_a^b f'(x) = f(b) - f(a)\]
	\section*{Integral como Masa dada Densidad}
	Si $\rho : [a,b] \to \R$ es la densidad de un alambre, luego $\int_a^b \rho(x) dx$ es la masa del alambre.

	\section*{Integral como Probabilidad}
	Si tengo $\rho(x) = \frac{1}{\sqrt{2\pi}}e^{-\frac{x^2}{2}}$ la campana de Gauss, luego para cualquier intervalo $[a,b]$, la probabilidad de que una variable aleatoria distribuida como $\rho$ caiga en $[a,b]$ es
	\[\int_a^b \rho(x) dx\]
\end{document}
