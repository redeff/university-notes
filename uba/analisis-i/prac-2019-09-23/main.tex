\documentclass{article}
\usepackage{amssymb}
\usepackage{mathtools}

\everymath{\displaystyle}
\setlength{\parskip}{3mm}
\setlength{\parindent}{0mm}

\def\R{\mathbb{R}}
\def\C{\mathbb{C}}
\def\N{\mathbb{N}}
\def\Q{\mathbb{Q}}

\date{}
\author{}

\begin{document}
\section*{Regla de la Cadena Multivariable}
Si tenemos $g : \R^p \to \R^n$ y $f : \R^n \to \R^m$, luego tenemos:


Se tiene
\[\D (f \circ g) x = \D f (gx) \times \D gx\]
Donde 
\[\D f : \R^n \to \R^{m \times n}\]
\[\D g : \R^p \to \R^{n \times p}\]
\[\D (f \circ g) : \R^p \to \R^{m \times p}\]
\section*{$\D$ bajo operaciones entre funciones}
Tenemos:
\begin{itemize}
	\item $\D(f+g) = \D f + \D g$
	\item $\D(f \times g) = \D f \times g + f \times \D g$
\end{itemize}

\section*{Ejercicio 1}
Tenemos:
\begin{itemize}
	\item $f(x,y,z) = (x^2+e^z, xy^2+z)$
	\item $gt = (\cos t, t^3, t^2+1)$
\end{itemize}
Calcular $\D (f \circ g)$.

Tenemos $(f \circ g)t = (\cos^2t + e^{t^2+1}, t^6\cos t + t^2 + 1)$, luego 
\[\D (f \circ g) = 
\begin{bmatrix}
	-2 \cos t \sin t + 2te^{t^2+1} \\
	-t^6 \sin t +6t^5 \cos t + 2t \\
\end{bmatrix}
\]

Alternativamente, tenemos:

\[
	\D f (x,y,z) = 
\begin{bmatrix}
	2x & 0 & e^z \\
	y^2 & 2xy & 1
\end{bmatrix}
\]
\[
	\D f(gt) = 
	\begin{bmatrix}
		2\cos t & 0 & e^{t^2+1} \\
		t^6 & 2t^3 \cos t & 1
	\end{bmatrix}
\]
\[
	\D gt = 
\begin{bmatrix}
	-\sin t \\
	3t^2 \\
	2t
\end{bmatrix}
\]

Luego:
\[
	\D (f\circ g) t = 
	\begin{bmatrix}
		2\cos t & 0 & e^{t^2+1} \\
		t^6 & 2t^3 \cos t & 1
	\end{bmatrix} \times
\begin{bmatrix}
	-\sin t \\
	3t^2 \\
	2t
\end{bmatrix} =
\]
\[
	\begin{bmatrix}
		-2 \sin t \cos t + 2te^{t^2+1} \\
		-t^6\sin t+ 6t^5 \cos t + 2t
		
	\end{bmatrix}
\]
Que da el mismo resultado.

\section*{Ejercicio 2}
Sean $f, g : \R^2 \to \R$, con $g(x,y) = e^{2x+y}f(6x^2+5xy+y^2, x+y+\sin(2x+y))$. Sabemos que $5x + 2y + z = 11$ es el plano tangente a $g$ en $(1, -2)$. Hallar el plano tangente a $f$ en $(0, -1)$.

Tenemos, por el plano tangente, que:
\[
	g(1, -2) = 11 - 5 \cdot 1 - 2 \cdot (-2) + 11 = -5 + 4 + 11 = 10
\]
Además, el gradiente será:
\[\D g(1, -2) = (-5, -2)\]
Sea $h (x,y) = (6x^2+5xy+y^2, x+y+\sin(2x+y))$, y sea $l = e^{2x+y}$.Luego tenemos
\[g = l \cdot (f \circ h)\]
\[\D g = l \cdot \D (f \circ h) + \D l \cdot (f \circ h)\]
\[\D g = l \cdot (\D f \circ h) \cdot \D h + \D l \cdot (f \circ h)\]
Si evalúo en $P = (1, -2)$, tenemos:
\[\D g P = lP \cdot \D f (hP) \cdot \D h P + \D l P \cdot f (h P)\]
\[\D g P = lP \cdot \D f (0, -1) \cdot \D h P + \D l P \cdot f (0, -1)\]
\[(-5, -2) = 1 \cdot \D f (0, -1) \cdot \D h P + \D l P \cdot 10\]
\[(-5, -2) = \D f (0, -1) \cdot \D h P + (2, 1) \cdot 10\]
Tenemos \[\D h =
	\begin{bmatrix}
		12x + 5y & 5x + 2y \\
		1 + 2\cos(2x+y) & 1 + \cos(2x+y)
	\end{bmatrix}
\]
Evaluado en $P$, tenemos
\[\D h P =
	\begin{bmatrix}
		2 & 1 \\
		3 & 2
	\end{bmatrix}
\]
Entonces tenemos:
\[(-5, -2) = \D f (0, -1) \cdot 
	\begin{bmatrix}
		2 & 1 \\
		3 & 2
	\end{bmatrix}
+ (2, 1) \cdot 10\]
\[(-25, -12) = \D f (0, -1) \cdot 
	\begin{bmatrix}
		2 & 1 \\
		3 & 2
	\end{bmatrix}
\]
Que es un sistema de ecuaciones.

\section*{Clase $\CC^k$}
Decimos que una función $f : (A \inn \R^n) \to \R$ es de clase $\CC^k$ si sus derivadas $k$-ésimas existen y son contínuas.

Todos los polinomios, senos y cosenos, exponenciales y sus sumas, restas, productos y divisiones (con denominador no nulo) son $\CC^\infty$

\section*{Demostrar que no es $\CC^2$}
tomemos $f(x,y) =
\begin{cases}
	\frac{x^3y}{x^2+y^2} & \text{Si $(x,y) \neq (0. 0)$} \\
	0 & \text{Si no}
\end{cases}$

Primero tenemos que calcular $f_x$ y $f_y$ en un entorno a $(0, 0)$, y nos queda:
\[
	f_x (x,y) =
	\begin{cases}
		\frac{x^4y+3x^2y^3}{(x^2+y^2)^2} & \text{Fuera del origen}\\
		0 & \text{En el origen}
	\end{cases}
\]
\[
	f_y (x,y) =
	\begin{cases}
		\frac{x^5-x^3y^2}{(x^2+y^2)^2} & \text{Fuera del origen}\\
		0 & \text{En el origen}
	\end{cases}
\]

Tenemos $f_{yx}(0, 0) = \lim_{t \to 0} \frac{f_y(t, 0) - f_y(0, 0)}{t} = \frac{t}{t} = 1$.

Y tenemos $f_{xy}(0, 0) = \lim_{t \to 0} \frac{f_x(0, t) - f_x(0, 0)}{t}$
\end{document}
